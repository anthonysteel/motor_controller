\documentclass[11pt]{article}

% Encoding and language
\usepackage[utf8]{inputenc}
\usepackage[T1]{fontenc}
\usepackage[english]{babel}

% Math and symbols
\usepackage{amsmath, 
            amssymb,
            graphicx,
            subcaption}
\usepackage[europeanresistors,americaninductors]{circuitikz}

% Better spacing and layout
\usepackage{microtype}
\usepackage{geometry}
\geometry{margin=1in}

% Hyperlinks
\usepackage[colorlinks=true, linkcolor=blue, urlcolor=blue, citecolor=blue]{hyperref}

\title{Permanent-Magnet Motor Control}
\author{Anthony J. Steel}
\date{\today}

\begin{document}

\maketitle
\section{Introduction}

The fundamental electromagnetic unit of a motor is the interaction of two fields
in a region called the \emph{air-gap}. The goal of motor control is to manipulate
the fields in the air-gap to control relevant physical quantities such as
torque, speed, position, energy, and power. In a permanent-magnet motor one of
these fields is produced by permanent magnets and the other is produced by
currents. The mechanism that contains the permanent magnets is called the
\emph{rotor} because it is free to rotate, while the one that carries the currents
is called the \emph{stator} because it is generally stationary relative to the
labratory.

\section{Preliminaries}

\subsection{Maxwell's Equations}
Begin with the macroscopic Maxwell equations in SI units,

\begin{align}
\nabla \cdot \mathbf{B} &= 0,\\
\nabla \times \mathbf{E} &= -\frac{\partial \mathbf{B}}{\partial t},\\
\nabla \cdot \mathbf{D} &= \rho_\mathrm{f},\\
\nabla \times \mathbf{H} &= \mathbf{J}_\mathrm{f} + 
\frac{\partial\mathbf{D}}{\partial t},
\end{align}
where $\rho_\mathrm{f}$ and $\mathbf{J}_\mathrm{f}$ denote the free charge and
free current densities, and
\[
\mathbf{B} = \mu_0(\mathbf{H} + \mathbf{M}),\qquad
\mathbf{D} = \varepsilon_0 \mathbf{E} + \mathbf{P}
\]
are the magnetic flux density and electric displacement expressed in terms of
the magnetic and electric polarizations $\mathbf{M}$ and $\mathbf{P}$.

\subsection{Magnetostatic Approximation}
The finite propogation of electromagnetic disturbances in the air-gap is
$c = 1/\sqrt{mu_0\varepsilon_0} \approx 3 \times 10^8\, \mathrm{m/s}$.
Let $g$ denote the characteristic thickness of the air-gap (typically of the 
order of $1\mathrm{mm}$) and let $f_e$ be the electrical excitation frequency of
the motor. The time required for an electromagnetic distrubance to cross the air-
gap is 
\[
\tau = \frac{1}{f_e}
\]
The ratio of these time scales is
\[
\varepsilon_t = \frac{\tau_{\mathrm{em}}}{T_e} = \frac{g f_e}{c}
\]
Even for conservative values such as $g = 1\,\mathrm{mm}$ and $f_e = 10\,\mathrm{kHz}$
one obtains 
\[
\varepsilon_t = \frac{10^{-3}\times 10^4}{3\times 10^8}\approx 3 \times 10^{-8}
\]
and for typical permanent-magnet motors operating in the $f_e \lesssim 10^3\,\mathrm{Hz}$
the ratio is several orders of magnitude smaller. Thus the electromagnetic field
configuration across the air-gap adjusts effectively instantaneously relative to
the mechanical and electrical time scales of the machine.

This seperation of time scales can also be expressed directly in Maxwell's
equations. In the current-free, homogenous air-gap the magnetic field satisfies
the homogenous wave equation
\[
\nabla^2 \mathbf{H} - 
\mu_0\varepsilon_0 \frac{\partial^2 \mathbf{H}}{\partial t^2} = 0
\]
Introducing characteristic length $L$ (here $L \sim g$) and angular frequency
$\omega = 2 \pi f_e$, and dimensionles variables
$\tilde{\mathbf{x}} = \mathbf{x}/L$ and $\tilde{t} = \omega t$, this equation 
takes the form
\[
\tilde{\nabla}^2\mathbf{H} - \left(\frac{\omega L}{c}\right)^2
\frac{\partial^2 \mathbf{H}}{\partial \tilde{t}^2} = 0.
\]
The dimensionless parameter 
\[
\varepsilon_\omega = \frac{\omega L}{c} = \frac{2\pi f_e L }{c}
\]
measures the importance of wave-propogation effects. For the dimensions and 
frequencies of interest in rotating electrical machines $\varepsilon_\omega \ll 1$
and therefore the wave term can be neglected. The field in the air-gap then
satisfies the magnetostatic equation
\[
\nabla \times \mathbf{H} \approx \mathbf{0}, \qquad 
\nabla \cdot \mathbf{B} = 0,
\]
with $\mathbf{B} = \mu_0 \mathbf{H}$.

\subsection{Air-gap}
The air-gap region $V_g$ is the annular region of space between the rotor and 
stator surfaces, as shown in Figure~\ref{fig:air-gap}.  On the macroscopic level
of Maxwell's equations, material effects enter through the polarization $\mathbf{P}$
and magnetization $\mathbf{M}$, which relate to the fields via
\[
\mathbf{D} = \varepsilon_0 \mathbf{E} + \mathbf{P}, \qquad 
\mathbf{B} = \mu_0(\mathbf{H} + \mathbf{M}).
\]
Free charges and currents are described by the free charge density $\rho_mathrm{f}$
and free current density $\mathbf{J}_\mathrm{f}$. Bound charges and currents 
associated with materials are encoded in $\mathbf{P}$ and $\mathbf{M}$, in particular,
the bound current density is $\mathbf{J}_\mathrm{b} = \nabla \times \mathbf{M}$ 
and the total macroscopic current density is $\mathbf{J}_\mathrm{f} + \mathrm{J}_\mathrm{b}$.

In electric machines, the sources of magnetic fields are
\begin{itemize}
    \item free currents flowing in the stator,
    \item bound currents associated with magnetized materials in the rotor's
          permanent magnets.
\end{itemize}
By construction, all current carrying conductors are embedded in the stator and 
do not cross the air-gap. Likewise, all magnetized materials are confined to the
rotor and do not cross the air-gap. The air-gap $V_g$ contains only air and
consequently
\[
\mathbf{J}_\mathrm{f} = \mathbf{0},\qquad \mathbf{M} = \mathbf{0}\quad
\text{in }V_g.
\]

The medium filling the gap is air at standard conditions. On the macroscopic
scale relevant to machines, air is extremely weakly polarizable and non-magnetizable,
with relative permittivity and permeability very close to linear. It is therefore
an excellent approximation to treat the air-gap as a linear, homogenous, isotropic
medium with
\[
\mathbf{P} \approx \mathbf{0}, \qquad \mathbf{M}\approx \mathbf{0},
\]
so that the constitutive relations reduce to 
\[
\mathbf{B} = \mu_0 \mathbf{H}, \qquad
\mathbf{D} = \varepsilon_0 \mathbf{E}
\quad \text{in } V_g
\]
In other words, within the air-gap the electromagnetic field behaves as in 
vacuum: there are no material sources, and all sources are effectively represented
by boundary conditions on the enclosing rotor and stator surfaces.

\subsection{Electromagnetic force, momentum, and the Maxwell stress tensor}
A rigid rotor experiences torque because electromagnetic fields exert mechanical
forces on the charges and currents it contains.  At the macroscopic level this
mechanical force density is given by the Lorentz expansion
\begin{equation}
\mathbf{f} = \rho\,\mathbf{E} + \mathbf{J}\times\mathbf{B},
\label{eq:lorentz-force-density}
\end{equation}
where $\rho$ and $\mathbf{J}$ are the total charge and current densities 
(with bound and free contributions). The total force on a body occupying a
volume $V$ is therefore
\begin{equation}
\mathbf{F} = \int_V \mathbf{f}\, dV,
\label{eq:force-volume-integral}
\end{equation}
and the total torque about the origin is
\begin{equation}
\boldsymbol{\tau} = \int_V \mathbf{r}\times\mathbf{f}\, dV.
\label{eq:torque-volume-integral}
\end{equation}
For a rotating electrical machine this volume would include the detailed
distributions of currents and materials in the rotor and stator. Evaluating
the \eqref{eq:force-volume-integral} -- \eqref{eq:torque-volume-integral} 
directly in terms of $\rho$ and $\mathbf{J}$ is inconvenient and obscures the 
fact that the net torque depends only on the electromagnetic field in the air-gap.

A more useful representation of the mechanical action of the field is obtained
by recasting the Lorentz force density in terms of the electromagnetic field
itself. Combining \eqref{eq:lorentz-force-density} with Maxwell's equations and
using standard vector identities one finds the local momentum-balance relation
\begin{equation}
\mathbf{f} = \nabla \cdot \mathbf{T}
- \frac{\partial \mathbf{g}}{\partial t}
\label{eq:momentum-balance}
\end{equation}
where
\begin{equation}
\mathbf{g} = \varepsilon_0\, \mathbf{E} \times \mathbf{B}
\label{eq:field-momentum-density}
\end{equation}
is the electromagnetic momentum density and $\mathbf{T}$ is the \emph{Maxwell
stress tensor} with Cartesian components
\begin{equation}
T_{ij} = \varepsilon_0\Bigl(E_i E_j - \tfrac{1}{2}\delta_{ij} E^2\Bigr)
+ \frac{1}{\mu_0}\Bigl(B_i B_j - \tfrac{1}{2}\delta_{ij}B^2\Bigr),
\qquad i, j \in \{x, y, z\}
\label{eq:maxwell-stress-general}
\end{equation}
Equation \eqref{eq:momentum-balance} expresses conservation of total momentum:
the Lorentz force density $\mathbf{f}$ acting on matter is balanced by the
divergence of the field stress and the time rate of change of the electromagnetic
momentum.

Integrating \eqref{eq:momentum-balance} over a volume $V$ and applying the
divergence theorem gives
\begin{equation}
\mathbf{F} = \int_V \mathbf{f}\, dV
= \oint_{\partial V} \mathbf{T} \cdot \mathbf{n}\, dS - 
\frac{d}{dt}\int_V \mathbf{g}\, dV,
\label{eq:force-surface-integral}
\end{equation}
where $\partial V$ is the boundary surface of $V$ with outward unit normal 
$\mathbf{n}$. Thus the net mechanical force on the material inside $V$ can
be computed directly from the electromagnetic field on the boundary of $V$.
A similar manipulation applied to $\eqref{eq:torque-volume-integral}$ yields
\begin{equation}
  \boldsymbol{\tau} =
  \oint_{\partial V} \mathbf{r}\times (\mathbf{T}\cdot\mathbf{n})\, dS
  -\frac{d}{dt}\int_V\mathbf{r}\times\mathbf{g}\, dV.
  \label{eq:torque-surface-integral}
\end{equation}
The vector $\mathbf{T}\cdot\mathbf{n}$ is the traction exerted by the field on
a surface element with normal $\mathbf{n}$: its normal component corresponds to
an electromagnetic pressure and its tangential components correspond to shear
stresses.

In the magnetoquasistatic operating regime of a permanent-magnet motor, the
fields in the air-gap vary slowly and the electric field is small compared with 
the magnetic field insofar as mechanical effects are concerned. To an excellent
approximation in the air-gap, the electric contribution to the stress tensor 
and field-momentum term may be neglected, so that 
\eqref{eq:maxwell-stress-general} reduces
\begin{equation}
T_{ij} = \frac{1}{\mu_0}\Bigl(B_i B_j - \tfrac{1}{2}\delta_{ij} B^2\Bigr),
\label{eq:maxwell-stress-magnetic}
\end{equation}
and the torque on the material inside $V$ can be obtained from the purely
magnetic surface integral
\begin{equation}
  \boldsymbol{\tau}
  = \oint_{\partial V} \mathbf{r} \times (\mathbf{T}\cdot\mathbf{n})\, dS.
  \label{eq:torque-magnetic-stress}
\end{equation}

The key advantage of \eqref{eq:torque-magnetic-stress} in the motor context is
that $\partial V$ may be taken to lie entirely in the current-free,
magnetization-free air-gap.  All the details of the stator and rotor implementation
enter only through the magnetic field $\mathbf{B}$ on this surface.  The 
Maxwell stress tensor therefore provides a rigorous way to compute the net 
electromagnetic torque on the rotor using only the air-gap field.

For the machines considered here the geometry is ideally invariant in the axial
$z$-direction and only torque about the $z$-axis is of interest.  A convenient
choice for $\partial V$ is a cylindrical surface of radius $R$ in the air-gap,
coaxial with the rotor and extending over the stack length $L$. On this surface
the outward normal is $\mathbf{n} = \hat{\mathbf{e}}_r$ and the traction is
\[
\mathbf{T}\cdot\mathbf{n} = 
T_{rr} \hat{\mathbf{e}}_r + T_{\theta r} \hat{\boldsymbol{\theta}} +
T_{zr} \hat{\mathbf{e}}_z
\]
The torque density about the $z$-axis is determined by the tangential component
of this traction:
\[
d\tau_z = R\, T_{\theta r}\, dS,
\]
with $dS = R L\, d\theta$ for the cylndrical surface element.  In the 
two-dimensional model used here the fields are independent of $z$ and $B_z\approx 0$,
so that \eqref{eq:maxwell-stress-magnetic} gives
\begin{equation}
T_{\theta r} = \frac{1}{\mu_0} B_\theta B_r.
\label{eq:t-theta-r}
\end{equation}
Integrating around the circumference yields the electromagnetic torque about the
$z$-axis,
\begin{equation}
  \tau_z
= \frac{R^2 L}{\mu_0} \int_0^{2\pi} B_r(\theta)\, B_\theta(\theta)\, d\theta,
\label{eq:torque-Br-Btheta}
\end{equation}
which will serve as the starting point for the harmonic analysis in the following
sections.  This expression shows explicitly that the torque on the rotor is
goverend by the interaction between the radial and tangential components of the
magnetic field in the air-gap.

\begin{figure}[h!]
    \centering
    \includegraphics[width=\textwidth]{plots/img/air_gap.png}
    \caption{The air-gap of a motor. The region of space where the stator and r
    otor fields interact to produce torque on the surface of the rotor. No
    fields are shown.}
    \label{fig:air-gap}
\end{figure}


\section{Solving Maxwell's Equations in the Air-Gap}
\subsection{Scalar Magnetic Potential and Laplace's Equation}
Within the air-gap rergion $V_g$ there are no free conduction currents and no
magnetized materials, and the medium is well approximated as a linear, homogenous
and isotropic. In the magnetoquasistatic regime this implies that Maxwell's equations
reduce in $V_g$ to
\begin{equation}
\nabla \times \mathbf{H} = \mathbf{0}, \qquad
\nabla \cdot \mathbf{B} = 0, \qquad \mathbf{B} = \mu_0 \mathbf{H}
\label{eq:magnetostatic-gap}
\end{equation}
The first of these shows that $\mathbf{H}$ is irrotational in the air-gap.
Hence, on any simple-connected subregion of $V_g$ there exists a scalar magnetic
potential $\Phi$ such that
\begin{equation}
\mathbf{H} = -\nabla\Phi \qquad \Rightarrow \qquad \mathbf{B} = -\mu_0\nabla\Phi.
\label{eq:B-from-Phi-gap}
\end{equation}
\begin{equation}
\nabla^2 \Phi = 0 \qquad \text{in } V_g,
\label{eq:laplace-gap}
\end{equation}
so the scalar magnetic potential is harmonic in the air-gap.

As previously discussed, in the machines considered here the fields are approximetly
invariant along the axial direction $z$, so the problem reduces to two dimensions
in the cross-sectional $(r,\theta)$ plane.  In cylindrical coordinates with no
$z$-dependence, the Laplace equation \eqref{eq:laplace-gap} becomes
\begin{equation}
\frac{\partial^2 \Phi}{\partial r^2}
+ \frac{1}{r}\frac{\partial \Phi}{\partial \theta^2} = 0.
\label{eq:laplace-polar}
\end{equation}
On the annular ring $R_s < r < R_r$ the general seperable solution of 
\eqref{eq:laplace-polar} can be written as
\begin{equation}
  \Phi(r,\theta) = A_0 + B_0\ln r + \sum_{n=1}^\infty \Bigl[
    \bigl(a_n r^n + b_n r^{-n}\bigr)\cos(n\theta) + \bigl(c_n r^n + d_n r^{-n}\bigr)
    \sin(n\theta)\Bigr],
\label{eq:Phi-general}
\end{equation}
where the coefficients are determined by the boundary conditions imposed by the
rotor and stator on $r=R_s$ and $r=R_r$.  The constant $A_0$ is an arbitary gauge
for the potential and does not affect the magnetic field.  The $B_0\ln r$ term
and the $r^{\pm n}$ terms represent different radial behaviours compatible with
Laplace's equation on the finite annulus; the particular combination present in
a given machine is fixed by the physical boundary conditions.

The magnetic flux density components in cylindrical coordinates are obtained
from \eqref{eq:B-from-Phi-gap} as
\begin{equation}
B_r(r,\theta) = -\mu_0 \frac{\partial \Phi}{\partial r}, \qquad
B_\theta(r, \theta) = -\frac{\mu_0}{r}\frac{\partial \Phi}{\partial \theta}.
\label{eq:BrBtheta-from-Phi}
\end{equation}
For the torque calculation it is sufficient to know $B_r$ and $B_\theta$ on a
single cylndrical surface of radius $R$ lying in the air-gap. At any fixed
radius $R$ the functions $B_r(R,\theta)$ and $B_\theta(R,\theta)$ are
$2\pi$-periodic in $\theta$ and therefore admit a Fourier series of the form
\begin{equation}
B_r(R,\theta) = \sum_{n=1}^\infty \bigl(a_n \cos(n\theta) + b_n\sin(n\theta)\bigr),
\end{equation}

\subsection{Ideal Rotor and Stator Fields}
Because the magnetostatic equations in the air-gap are linear, the total magnetic 
field can be decomposed into contributions from the rotor and stator sources.
Let $\mathbf{B}^r$ denote the field produced by the rotor alone with the
stator source currents set to zero, and let $\mathbf{B}^2$ denote the field 
produced by the stator currents with the rotor demagnetized.  In the air-gap
$V_g$ both fields satisfy the same governing equations,
\begin{equation}
\nabla\times\mathbf{H}^{r,s} = \mathbf{0},\qquad
\nabla\cdot\mathbf{B}^{r,s} = 0,\qquad
\mathbf{B}^{r,s} = \mu_0 \mathbf{H}^{r,s},
\end{equation}
and superposition gives
\begin{equation}
\mathbf{B} = \mathbf{B}^r + \mathbf{B}^s,\qquad
\mathbf{H} = \mathbf{H}^r + \mathbf{H}^s.
\end{equation}
The distinct role of the rotor and stator enter through their boundary conditions 
at the rotor surface $r=R_r$ and stator surfaces $r=R_s$, respectively.
These boundary conditions are determined by the internal magnetization and currents
in the rotor and stator regions.

At any material interface the fields satisfy
\begin{align}
  \hat{\mathbf{n}}\cdot(\mathbf{B}_2 - \mathbf{B}_1) &= 0, \label {eq:BC-normalB}\\
  \hat{\mathbf{n}}\times(\mathbf{H}_2 -\mathbf{H}_1) &= \mathbf{H}_\mathrm{f},
  \label{eq:BC-tangH}
\end{align}
where $\hat{\mathbf{n}}$ is the unit normal pointing from region $1$ into region
$2$ and $\mathbf{K}_\mathrm{f}$ is any free surface current density on the interface.
Equation \eqref{eq:BC-normalB} expresses continuity of the normal component of
$\mathbf{B}$, while \eqref{eq:BC-tangH} relates the jump in tangential $\mathbf{H}$
to surface currents. In a typical machine the rotor and stator iron have relative
permeability $\mu_r \gg 1$, so that, to a good approximation, $\mathbf{B}$ is nearly
normal to the iron surfaces and the normal component is determined by the internal
magnetization or current distribution.

\subsection{Ideal Rotor}
Consider the rotor assembly occupying $r < R_r$, consisting of high-permeability
iron and permanent magnets. With the stator unexcited, the rotor field $\mathbf{B}^r$
in the air-gap is governed by the magnetostatic equations with boundary conditions on
$r = R_r$ fixed by the rotor magnetization.  An \emph{ideal rotor of spatial order
$n$} is defined as a rotor for which, in this rotor-only configuration, the radial
and tangential components of the air-gap field at the surface of the rotor satisfy
\begin{equation}
  B_r^r(R_r,\theta) = a \cos(n\theta),\qquad
  B_\theta^r(R_r, \theta) = 0,
  \label{eq:ideal-rotor-BC}
\end{equation}
for some real amplitude $a > 0$ and integer $n \ge 1$.

This definition encodes two simplifying properties:
\begin{enumerate}
  \item The radial flux on the rotor surface is a \emph{single spatial harmonic}
        of order $n$ (no higher-order harmonics are present).
  \item The field at the rotor surface is \emph{purely radial} (normal to the
        surface), consistent with the limit of very high rotor permeability in
        which the tangential component of $\mathbf{B}$ at the iron surface is 
        supressed.
\end{enumerate}
Physically, such a field can be realized arbitrarily closely by an appropriate
circumferential magnetization pattern in a cylindrical permanent-magnet shell
backed by a high permeability iron, as illustrated in Figure \ref{fig:rotor}.
The constant $n$ is the number of \emph{pole pairs} $p$ of the rotor field.

\subsection{Ideal Stator}
Similarly, consider the stator assembly occuping $r > R_s$, consisting of 
ferromagnetic iron and current-carrying windings. With the rotor demagnetized,
the stator field $\mathbf{B}^s$ in the air-gap is determined by Ampere's Law
and the current distribution in the windings. The tangential component of $\mathbf{H}^s$
at the stator inner surface is related to the stator surface current density
$\mathbf{K}_\mathrm{f}$ by \eqref{eq:BC-tangH}. In many machines the stator is
designed so that the resulting air-gap field is dominated by a single spatial
harmonic of perscribed order.

An \emph{ideal stator of spatial order $n$} is defined as a stator for which,
in the stator-only configuration, the air-gap field at the stator surface satisfied
\begin{equation}
B_r^s(R_s,\theta) = 0, \qquad 
B_\theta^s(R_s,\theta) = b\cos(n\theta - \phi_s),
\label{eq:ideal-stator-BC}
\end{equation}
for some amplitude $b>0$, phase $\phi_s$, and the same integer $n \ge 1$.
Here 
\begin{enumerate}
  \item The tangential flux density on the stator surface is a \emph{single
  spatial harmonic} of order $n$, corresponding to a sinusoidally distributed 
  stator magnetomotive force (MMF) with $n$ pole pairs.
  \item The field at the stator surface is \emph{purely tangential}, as expected
  in the ideal limit of a smooth, high-permeability stator core with no normal
  flux crossing the inner surface except through discrete teeth.
\end{enumerate}
In practice, a distributed stator winding with appropriately chosen coil pitch
and distribution can approximate \eqref{eq:ideal-stator-BC} closely by supressing
higher spatial harmonics and producing a nearly sinusoidal rotating air-gap field.

\subsection{Ideal Rotor-Stator Pair and Torque}
When both the rotor and stator are present and energized, the total air-gap field
is the sum
\[
B_r = B_r^r + B_r^s, \qquad
B_\theta = B_\theta^r + B_\theta^s.
\]
For an ideal rotor--stator pair of order $n$, as defined above, the dominant
contribution to the torque arises from the interaction of the $n$th spatial
harmonic of $B_r$ and $B_\theta$. Evaluated on any cylindrical surface 
$R_r < R < R_s$ in the air-gap, these harmonics inherit the same spatial order
$n$ and their amplitudes and phase can be written as
\[
B_r(R,\theta) \approx B_{r,n}\cos(n\theta - \phi_r),\qquad
B_\theta(R,\theta)\approx B_{\theta,n}\cos(n\theta - \phi_\theta),
\]
where $B_{r,n}$ and $B_{\theta,n}$ are proportional to $a$ and $b$ and $\phi_r$
and $\phi_\theta$ are the corresponding phase angles. Substituting these into
the general torque expression
\[
T = \frac{\pi R^2 L}{\mu_0} \sum_{k=1}^\infty B_{r,k} B_{\theta,k} \cos\bigl(
  \phi_{r,k} - \phi_{\theta,k}\bigr),
\]
For a perfectly aligned ideal rotor--stator pair we choose the phase reference
such that $\phi_r = \phi_\theta$, so that the $n$th-order contribution is maximal
and
\begin{equation}
T_\mathrm{max} = \frac{\pi R^2 L}{\mu_0} B_{r,n} B_{\theta, n}.
\end{equation}
In particular, evaluated on the rotor surface $R = R_r$ for the boundary conditions 
\eqref{eq:ideal-rotor-BC}--\eqref{eq:ideal-stator-BC} with $\phi_s=0$, one has
$B_{r,n}=a$ and $B_{\theta,n}=b$ and therefore
\begin{equation}
  T_\mathrm{max} = \frac{\pi R^2 L}{\mu_0}\, a b.
  \label{eq:T-ideal-ab}
\end{equation}
The ideal rotor and stator thus provide a useful reference: for a given air-gap
radius and stack length, they define the maximum torque obtainable from a single
spatial harmonic of radial and tangential flux density in the air-gap.

\subsection{Why the stator field must rotate for continuous torque}

The harmonic torque expression \eqref{eq:T-harmonic-amplitude-phase} shows that
the electromagnetic torque depends on the amplitudes and relative phases of the
rotor- and stator-generated field harmonics in the air-gap. For a single
dominant spatial order $n$ this expression reduces to
\begin{equation}
T(t)
= K\, B_{r,n}(t) B_{\theta,n}(t)
    \cos\bigl(\phi_{r,n}(t) - \phi_{\theta,n}(t)\bigr),
\label{eq:T-single-harmonic-time}
\end{equation}
where
\[
K = \frac{\pi R^2 L}{\mu_0},
\]
$B_{r,n}(t)$ and $B_{\theta,n}(t)$ are the instantaneous amplitudes of the
$n$th radial and tangential harmonics, and
$\phi_{r,n}(t)$, $\phi_{\theta,n}(t)$ are their instantaneous spatial phases
with respect to a fixed reference axis on the stator.

For an ideal permanent-magnet rotor of order $n$ the radial component of the
rotor field at a fixed radius $R$ can be written as a traveling wave in the
stator reference frame,
\begin{equation}
B_r^r(R,\theta,t)
= \widehat{B}_r \cos\bigl(n(\theta - \theta_m(t))\bigr),
\label{eq:Br-rotor-traveling}
\end{equation}
where $\theta_m(t)$ is the mechanical rotor angle and $n$ is the number of pole
pairs. At any fixed time $t$, comparing \eqref{eq:Br-rotor-traveling} with the
general form
\[
B_r(R,\theta,t)
= B_{r,n}(t)\cos\bigl(n\theta - \phi_{r,n}(t)\bigr)
\]
shows that
\begin{equation}
B_{r,n}(t) = \widehat{B}_r, \qquad
\phi_{r,n}(t) = n\,\theta_m(t) + \phi_{r,0},
\label{eq:phi-r-time}
\end{equation}
for some constant phase offset $\phi_{r,0}$ that depends on the choice of
reference axis.

Similarly, the $n$th-order component of the stator field at the same radius
$R$ can be written as
\begin{equation}
B_\theta^s(R,\theta,t)
= B_{\theta,n}(t)\cos\bigl(n\theta - \phi_{\theta,n}(t)\bigr),
\label{eq:Btheta-stator-general}
\end{equation}
where $B_{\theta,n}(t)$ and $\phi_{\theta,n}(t)$ are determined by the stator
currents. It is convenient to interpret $\phi_{\theta,n}(t)$ as the
\emph{instantaneous angular position} of the stator field pattern of order $n$:
the peaks of $B_\theta^s$ occur at angles $\theta$ satisfying
$n\theta - \phi_{\theta,n}(t) = 0$, i.e.
\[
\theta = \frac{\phi_{\theta,n}(t)}{n}.
\]
Thus the quantity
\begin{equation}
\theta_s(t) := \frac{\phi_{\theta,n}(t)}{n}
\label{eq:theta-s-def}
\end{equation}
can be regarded as the stator field angle for the $n$th harmonic.

Substituting \eqref{eq:phi-r-time} and \eqref{eq:theta-s-def} into
\eqref{eq:T-single-harmonic-time} gives
\begin{equation}
T(t)
= K\,\widehat{B}_r B_{\theta,n}(t)
  \cos\bigl(n\theta_m(t) - n\theta_s(t) + \Delta\phi_0\bigr),
\label{eq:T-delta-theta}
\end{equation}
where $\Delta\phi_0$ is a constant phase offset. Defining the instantaneous
\emph{electrical torque angle}
\begin{equation}
\delta(t)
:= n\bigl(\theta_m(t) - \theta_s(t)\bigr) + \Delta\phi_0,
\label{eq:delta-def}
\end{equation}
we can write
\begin{equation}
T(t)
= K\,\widehat{B}_r B_{\theta,n}(t)\cos\delta(t).
\label{eq:T-cos-delta}
\end{equation}
This expression shows that, for given amplitudes, the torque is maximized when
$\delta(t)$ is held at a constant value near $0$ or $\pi$ (depending on the
desired direction of torque) and vanishes when the fields are in spatial
quadrature, $\delta = \pm \tfrac{\pi}{2}$.

To understand the time evolution of $\delta(t)$, differentiate
\eqref{eq:delta-def}:
\begin{equation}
\dot{\delta}(t)
= n\bigl(\dot{\theta}_m(t) - \dot{\theta}_s(t)\bigr)
= n\bigl(\omega_m(t) - \omega_s(t)\bigr),
\label{eq:delta-dot}
\end{equation}
where
\[
\omega_m(t) = \dot{\theta}_m(t)
\]
is the mechanical angular velocity of the rotor and
\[
\omega_s(t) = \dot{\theta}_s(t)
\]
is the angular velocity of the stator field pattern (both measured in the
stator reference frame). Equation \eqref{eq:delta-dot} makes the key point:
\begin{itemize}
\item If $\omega_s(t) = \omega_m(t)$, then $\dot{\delta}(t) = 0$ and the
      torque angle $\delta$ is constant in time. In this case the rotor
      experiences a \emph{steady} torque given by \eqref{eq:T-cos-delta}.
\item If $\omega_s(t) \neq \omega_m(t)$, then $\delta(t)$ drifts in time at
      the slip rate $n(\omega_m - \omega_s)$, and the torque oscillates as
      $\cos\delta(t)$ sweeps between positive and negative values.
\end{itemize}

As a simple illustration, suppose the stator field is \emph{stationary} in the
stator frame, so that $\omega_s(t) = 0$ and $\theta_s(t) = \theta_{s,0}$ is
constant. Then
\[
\delta(t)
= n\bigl(\theta_m(t) - \theta_{s,0}\bigr) + \Delta\phi_0.
\]
If we start the rotor from rest and apply this static stator field, the initial
torque tends to accelerate the rotor toward alignment with the stator field.
As $\theta_m(t)$ increases, the angle $\delta(t)$ passes through zero and
continues to grow in magnitude; once $|\delta| > \tfrac{\pi}{2}$ the torque
reverses sign and tends to decelerate the rotor. In the absence of time-varying
control of the stator currents, the rotor settles into a static equilibrium
with its magnetic axis aligned to the stator field, and the torque averages to
zero over time. A purely spatially fixed stator field can therefore produce at
most a transient torque, not sustained rotation.

By contrast, if the stator currents are controlled so that the stator field
pattern rotates with angular velocity $\omega_s(t)$ chosen to satisfy
\[
\omega_s(t) \approx \omega_m(t),
\]
then $\dot{\delta}(t)$ can be made small or zero and the torque angle
$\delta(t)$ can be maintained near a desired constant value. In the ideal case
of a synchronous permanent-magnet machine with negligible dynamics, the stator
field is driven at a constant electrical angular frequency $\omega_e$, which
produces a rotating $n$th-order field pattern with angular velocity
\begin{equation}
\omega_s = \frac{\omega_e}{n}.
\end{equation}
The condition for a constant torque angle is then
\begin{equation}
\omega_e = n\,\omega_m,
\end{equation}
the familiar synchronous-speed relation. In this regime the rotor magnetic
pattern is effectively locked to the rotating stator field with a fixed angle
$\delta$, and the torque \eqref{eq:T-cos-delta} becomes a nonzero constant in
time.

In summary, the requirement of a rotating stator field follows directly from
the harmonic torque expression: the torque depends on the spatial phase
difference between rotor and stator field patterns; a static stator field
causes this phase difference to sweep through all angles as the rotor turns,
leading to alternating acceleration and deceleration and zero average torque.
To sustain a nonzero torque, the stator field must rotate in the stator frame
so that its angular velocity matches that of the rotor field and the torque
angle remains approximately constant.

\section{Stator Currents, Air-Gap Field, and Energy Transfer}
The torque expressions derived previously depend only on the magnetic field in
the air-gap.  In a permanent-magnet motor this field is produced jointly by the
rotor permanent magnets and the free-current distribution in the stator.  The 
purpose of this section is two-fold: (i) to connect the tangential air-gap field 
to the current distribution in the stator, and (ii) to show that the mechanical
power associated with the air-gap torque ultimately comes from the electrical
power supplied to the stator currents.

\subsection{Stator current distribution and the air-gap field}
Let $V_s$ denote the stator region containing the windings and iron, and let
$\mathbf{J}_\mathrm{f}^s(\mathbf{r})$ be the free current density in the stator
contributions. In magnetostatics the fields generated by these currents satisfy
\begin{equation}
\nabla\times\mathbf{H}^s = \mathbf{J}_\mathrm{f}^s, \qquad
\nabla\cdot\mathbf{B}^s = 0, \qquad
\mathbf{B}^s = \mu_0 \mathbf{H}^s \quad \text{in } V_g,
\label{eq:stator-field-eqs}
\end{equation}
with appropriate constitutive relations in the stator iron. The superscript
$s$ emphasizes that these are the fields that would exist if the rotor
magnets were removed or demagnetized. Linearity then gives the total fields as
\[
\mathbf{B} = \mathbf{B}^r + \mathbf{B}^s, \qquad
\mathbf{H} = \mathbf{H}^r + \mathbf{H}^s.
\]To relate $\mathbf{J}_\mathrm{f}^s$ to the air-gap field, we consider the
interface between the stator and the air-gap. Let $S_s$ be the inner surface
of the stator at $r = R_s$, with unit normal $\hat{\mathbf{n}}$ pointing into
the air-gap. The macroscopic boundary conditions at this interface are
\begin{align}
\hat{\mathbf{n}}\cdot(\mathbf{B}_g - \mathbf{B}_s) &= 0,
\label{eq:BC-stator-normalB} \\
\hat{\mathbf{n}}\times(\mathbf{H}_g - \mathbf{H}_s) &= \mathbf{K}_\mathrm{f},
\label{eq:BC-stator-tangH}
\end{align}
where subscripts $g$ and $s$ denote quantities evaluated just inside the
air-gap and stator, respectively, and $\mathbf{K}_\mathrm{f}$ is any free
surface current density flowing on $S_s$. Physically, $\mathbf{K}_\mathrm{f}$
represents the net current in the stator conductors crossing the interface per
unit length, and is related to the volume current density in the slots by
\[
\mathbf{K}_\mathrm{f}(\theta)
= \int_{\text{slot depth}} \mathbf{J}_\mathrm{f}^s(r,\theta)\, dr.
\]

In a typical machine the stator core is made of high-permeability iron, with
relative permeability $\mu_r \gg 1$. In this limit the tangential component of
$\mathbf{H}$ inside the stator iron is small compared with that in the
air-gap. Approximating $\mathbf{H}_s$ as negligible in the tangential
direction at $S_s$, \eqref{eq:BC-stator-tangH} gives
\begin{equation}
\hat{\mathbf{n}}\times\mathbf{H}_g \approx \mathbf{K}_\mathrm{f}.
\end{equation}
On the cylindrical surface $r = R_s$ with outward normal
$\hat{\mathbf{n}} = \hat{\mathbf{e}}_r$, this becomes
\begin{equation}
H_\theta^s(R_s,\theta) \approx K_\mathrm{f}(\theta),
\label{eq:Htheta-Kf}
\end{equation}
where $K_\mathrm{f}(\theta)$ is the circumferential component of the surface
current density. Using $\mathbf{B}^s = \mu_0\mathbf{H}^s$ in the air-gap, the
tangential component of the stator field at the air-gap boundary is therefore
\begin{equation}
B_\theta^s(R_s,\theta)
\approx \mu_0 K_\mathrm{f}(\theta).
\label{eq:Btheta-Kf}
\end{equation}

The surface current density $K_\mathrm{f}(\theta)$ is determined by the
stator winding geometry and the phase currents. For an arbitrary winding and
current distribution, $K_\mathrm{f}(\theta)$ can be expanded in a Fourier
series,
\begin{equation}
K_\mathrm{f}(\theta)
= \sum_{n=1}^\infty \bigl(K_n \cos(n\theta) + \tilde{K}_n \sin(n\theta)\bigr),
\end{equation}
and equations \eqref{eq:Htheta-Kf}--\eqref{eq:Btheta-Kf} show that the
resulting tangential air-gap field components inherit the same harmonic
structure:
\begin{equation}
B_\theta^s(R_s,\theta)
= \sum_{n=1}^\infty \bigl(\mu_0 K_n \cos(n\theta) + \mu_0 \tilde{K}_n \sin(n\theta)\bigr).
\label{eq:Btheta-Fourier-K}
\end{equation}
Comparing \eqref{eq:Btheta-Fourier-K} with the general expansion
\eqref{eq:BrBtheta-Fourier} shows that the Fourier coefficients
$\{c_n,d_n\}$ of the stator-generated $B_\theta$ are linear functions of the
Fourier coefficients $\{K_n,\tilde{K}_n\}$ of the stator current sheet, which
in turn are linear functions of the stator phase currents. Thus the tangential
air-gap field responsible for torque is directly and linearly controlled by
the stator currents.
\subsection{Energy transfer from stator currents to rotor torque}

The torque expression
\begin{equation}
T
= \frac{R^2 L}{\mu_0}\int_0^{2\pi} B_r(\theta)\,B_\theta(\theta)\,d\theta,
\end{equation}
obtained from the Maxwell stress tensor, determines the instantaneous
mechanical moment applied by the electromagnetic field to the rotor. At rotor
mechanical angular velocity $\omega_m$, the corresponding instantaneous
mechanical power delivered to the rotor is
\begin{equation}
P_\mathrm{mech} = T\,\omega_m.
\end{equation}
We now show that this mechanical power originates from the work done by the
electric field on the stator currents, i.e.\ from the electrical power supplied
to the windings.

The starting point is Poynting's theorem, which expresses local conservation
of electromagnetic energy:
\begin{equation}
\frac{\partial u}{\partial t}
+ \nabla\cdot\mathbf{S}
+ \mathbf{J}\cdot\mathbf{E} = 0.
\label{eq:Poynting-local}
\end{equation}
Here
\[
u = \frac{1}{2}\bigl(\mathbf{E}\cdot\mathbf{D}
                   + \mathbf{B}\cdot\mathbf{H}\bigr)
\]
is the electromagnetic energy density and
\[
\mathbf{S} = \mathbf{E}\times\mathbf{H}
\]
is the Poynting vector, representing the flux of electromagnetic energy. The
term $\mathbf{J}\cdot\mathbf{E}$ is the power density transferred from the
field to the charges: where $\mathbf{J}\cdot\mathbf{E}>0$, the field does work
on the charges; where $\mathbf{J}\cdot\mathbf{E}<0$, the charges do work on the
field.

Integrating \eqref{eq:Poynting-local} over a volume $V$ containing the stator,
air-gap and rotor, and applying the divergence theorem gives
\begin{equation}
\frac{d}{dt}\int_V u\, dV
+ \oint_{\partial V} \mathbf{S}\cdot\mathbf{n}\, dS
+ \int_V \mathbf{J}\cdot\mathbf{E}\, dV = 0.
\label{eq:Poynting-integral}
\end{equation}
The volume integral of $\mathbf{J}\cdot\mathbf{E}$ can be split into
contributions from the stator conductors, rotor conductors (if any) and the
rest of the machine. In a permanent-magnet motor without rotor windings, the
dominant term is the work done on the stator currents:
\[
P_\mathrm{elec}^\mathrm{stator}
= -\int_{V_s} \mathbf{J}_\mathrm{f}^s\cdot\mathbf{E}\, dV,
\]
which is precisely the electrical power delivered to the stator windings by
the external supply. The surface integral over $\partial V$ represents
electromagnetic radiation leaving the volume; for a rotating machine in the
magnetoquasistatic regime this term is negligible. Equation
\eqref{eq:Poynting-integral} then expresses the balance
\begin{equation}
P_\mathrm{elec}^\mathrm{stator}
= \frac{d}{dt}\int_V u\, dV + P_\mathrm{field\to mech} + P_\mathrm{loss},
\end{equation}
where $P_\mathrm{field\to mech}$ is the net rate at which the field does
mechanical work on the material bodies in $V$ (including the rotor and stator
iron), and $P_\mathrm{loss}$ accounts for ohmic and core losses. In particular,
the mechanical power transferred to the rotor is
\begin{equation}
P_\mathrm{mech}^\mathrm{rotor}
= T\,\omega_m,
\end{equation}
as obtained from the product of torque and angular velocity.

In steady-state operation the electromagnetic energy stored in the fields over
a cycle is periodic, so the time-average of $d/dt\int_V u\,dV$ vanishes. The
time-averaged Poynting balance then reduces to
\begin{equation}
\bigl\langle P_\mathrm{elec}^\mathrm{stator} \bigr\rangle
= \bigl\langle P_\mathrm{mech}^\mathrm{rotor} \bigr\rangle
 + \bigl\langle P_\mathrm{loss} \bigr\rangle,
\end{equation}
showing that, on average, the mechanical power associated with the air-gap
torque is drawn entirely from the electrical power supplied to the stator
currents, aside from losses. The role of the air-gap field is to mediate this
energy transfer: the stator currents establish the tangential field
$B_\theta^s$ in the air-gap, which interacts with the rotor field $B_r^r$ to
produce torque via the Maxwell stress, and the corresponding mechanical power
$T\omega_m$ is supplied by the electrical sources driving the stator currents.
\subsection{Flux linkage as a field--circuit coupling variable}

Up to this point the analysis has been formulated entirely in terms of
distributed electromagnetic fields in the air-gap. In order to connect this
field description to the electrical behavior of the stator, we need a
macroscopic quantity that couples the stator currents and voltages to the
magnetic field. This quantity is the \emph{flux linkage}. In this subsection we
introduce flux linkage directly from Maxwell's equations, without relying on
any particular realization of the stator windings.

\paragraph{Faraday's law for an abstract circuit.}
Consider an arbitrary closed conducting path $C$ embedded in the stator region
and moving rigidly with the stator (which is fixed in space). Let $S$ be any
oriented surface whose boundary is $C$, i.e.\ $\partial S = C$, and let
$\hat{\mathbf{n}}$ be the unit normal on $S$. Maxwell--Faraday's equation in
integral form states that the electromotive force (EMF) around $C$ is
\begin{equation}
\mathcal{E}
= \oint_C \mathbf{E}\cdot d\boldsymbol{\ell}
= -\frac{d}{dt} \int_S \mathbf{B}\cdot\hat{\mathbf{n}}\, dS.
\label{eq:Faraday-abstract}
\end{equation}
The surface integral
\begin{equation}
\Phi(t) := \int_S \mathbf{B}(\mathbf{r},t)\cdot\hat{\mathbf{n}}\, dS
\label{eq:flux-S}
\end{equation}
is the \emph{magnetic flux} through $C$. The value of $\Phi(t)$ is independent
of the particular choice of surface $S$ spanning $C$, provided $S$ lies in a
region where $\nabla\cdot\mathbf{B} = 0$, as is the case in the air-gap and
stator. Equation \eqref{eq:Faraday-abstract} may therefore be written simply
as
\begin{equation}
\mathcal{E}(t) = -\frac{d\Phi(t)}{dt}.
\label{eq:emf-dPhi-dt}
\end{equation}

In the macroscopic circuit description of the stator, the path $C$ will later
be associated with one \emph{stator circuit} or \emph{phase}, and the EMF
$\mathcal{E}(t)$ will be identified with the corresponding terminal voltage
(up to the sign convention adopted for voltage and current). At this stage,
however, we keep the description abstract: $C$ is simply a closed conducting
loop rigidly attached to the stator, and $\Phi(t)$ is the total magnetic flux
passing through it.

\paragraph{Definition of flux linkage.}
For a single isolated loop $C$ the magnetic flux $\Phi(t)$ defined in
\eqref{eq:flux-S} already provides a complete description of the coupling
between the loop and the field: Faraday's law \eqref{eq:emf-dPhi-dt} shows that
changes in $\Phi$ directly determine the induced EMF. In a real stator,
however, the conductors form extended paths that may pass through regions of
nonzero magnetic field multiple times. It is therefore convenient to introduce
the \emph{flux linkage} $\lambda(t)$ as the field quantity whose time
derivative gives the induced EMF in the corresponding stator circuit:
\begin{equation}
e(t) := \mathcal{E}(t) = -\frac{d\lambda(t)}{dt}.
\label{eq:emf-flux-linkage-def}
\end{equation}
For a single simple loop fixed in space one may simply set
$\lambda(t) = \Phi(t)$. For more complicated conductor geometries, $\lambda(t)$
is naturally interpreted as a linear functional of the magnetic field,
\begin{equation}
\lambda(t) = \mathcal{L}[\mathbf{B}(\cdot,t)],
\label{eq:lambda-functional}
\end{equation}
where the operator $\mathcal{L}$ encodes the geometry and connectivity of the
stator circuit under consideration. The only properties of $\mathcal{L}$ that
will be needed in the following are:
\begin{enumerate}
\item \emph{Linearity in $\mathbf{B}$}: if $\mathbf{B} = \mathbf{B}_1 +
      \mathbf{B}_2$ then
      $\lambda = \mathcal{L}[\mathbf{B}_1] + \mathcal{L}[\mathbf{B}_2]$;
\item \emph{Compatibility with Faraday's law}: the induced EMF in the circuit
      is given by \eqref{eq:emf-flux-linkage-def}.
\end{enumerate}
The detailed form of $\mathcal{L}$ will be specified later when particular
stator winding arrangements are introduced. For now, $\lambda$ may be regarded
as an abstract field-dependent state variable associated with each stator
circuit, defined so that $-d\lambda/dt$ equals the EMF induced by the changing
magnetic field.

\paragraph{Dependence on stator currents and rotor position.}
In the magnetoquasistatic regime the magnetic field $\mathbf{B}(\mathbf{r},t)$
is determined by the instantaneous stator currents and rotor position. Let
$i_1,\dots,i_m$ denote $m$ independent stator circuit currents and let
$\theta_m$ denote the rotor mechanical angle. Then, at each fixed time $t$, the
magnetic field in the machine can be written schematically as
\[
\mathbf{B}(\mathbf{r},t) = \mathbf{B}(\mathbf{r};\, i_1(t),\dots,i_m(t),\theta_m(t)).
\]
Correspondingly, the flux linkage associated with a given stator circuit $k$
can be regarded as a function of the instantaneous currents and rotor angle,
\begin{equation}
\lambda_k(t)
= \lambda_k\bigl(i_1(t),\dots,i_m(t),\theta_m(t)\bigr).
\label{eq:lambda-of-i-theta}
\end{equation}
Taking the time derivative and applying the chain rule yields
\begin{equation}
\frac{d\lambda_k}{dt}
= \sum_{\ell=1}^m
  \frac{\partial \lambda_k}{\partial i_\ell} \frac{di_\ell}{dt}
 + \frac{\partial \lambda_k}{\partial \theta_m} \frac{d\theta_m}{dt}.
\label{eq:lambda-chain-rule}
\end{equation}
Inserting this into Faraday's relation \eqref{eq:emf-flux-linkage-def} gives
the induced EMF in circuit $k$ as
\begin{equation}
e_k(t)
= -\sum_{\ell=1}^m
    \frac{\partial \lambda_k}{\partial i_\ell} \frac{di_\ell}{dt}
  - \frac{\partial \lambda_k}{\partial \theta_m} \omega_m(t),
\qquad
\omega_m = \frac{d\theta_m}{dt}.
\label{eq:emf-separation}
\end{equation}
The first term represents EMF associated with time variation of the currents,
i.e.\ the usual inductive effects. The second term is the EMF associated with
rotor motion through the magnetic field; this is the \emph{motional EMF} or
back-EMF that will play a central role in the rotor dynamics and torque--speed
behavior of the machine.

Equation \eqref{eq:emf-separation} shows that flux linkage is the natural
quantity that couples the distributed magnetic field to the lumped stator
circuit variables: its dependence on the currents determines inductive
behavior, while its dependence on rotor position determines the back-EMF and,
as will be seen, the electromagnetic torque.

\paragraph{Topological deformation and multiple flux linkages.}

The definition of flux linkage in \eqref{eq:emf-flux-linkage-def} arose from a
single closed conducting path $C$ and an associated magnetic flux $\Phi(t)$
through any surface $S$ with boundary $\partial S = C$. For an elementary loop,
it is natural to set $\lambda(t) = \Phi(t)$. In a real stator, however, the
conducting path may be deformed in space so that it passes through the same
magnetic field region multiple times. This deformation does not change the
fact that we have \emph{one} continuous circuit, but it does change how many
times the circuit ``links'' the magnetic flux. The effect of such a
deformation can be understood directly from Faraday's law.

Consider an initial simple loop $C_0$ that bounds a surface $S_0$ of area $A$
and is threaded by a magnetic field $\mathbf{B}(\mathbf{r},t)$. The flux
through $C_0$ is
\begin{equation}
\Phi_0(t) = \int_{S_0} \mathbf{B}\cdot\hat{\mathbf{n}}\, dS,
\end{equation}
and the EMF around $C_0$ is, by Faraday's law,
\begin{equation}
\mathcal{E}_0(t) = -\frac{d\Phi_0(t)}{dt}.
\end{equation}

Now imagine continuously deforming the conducting path $C_0$ into a new path
$C$ that winds around the same flux region $N$ times, as in a tightly coiled
conductor. Topologically, $C$ is still a single closed loop, but it can be
conveniently regarded as the series connection of $N$ simpler subloops
$C_1,\dots,C_N$, each of which threads essentially the same magnetic field
region. Formally, we can write the line integral of the electric field around
$C$ as the sum of line integrals around the subloops:
\begin{equation}
\oint_C \mathbf{E}\cdot d\boldsymbol{\ell}
= \sum_{k=1}^N \oint_{C_k} \mathbf{E}\cdot d\boldsymbol{\ell}.
\label{eq:emf-sum-subloops}
\end{equation}
Applying Faraday's law to each $C_k$ gives
\begin{equation}
\oint_{C_k} \mathbf{E}\cdot d\boldsymbol{\ell}
= -\frac{d\Phi_k(t)}{dt},
\end{equation}
where $\Phi_k(t)$ is the flux through any surface $S_k$ with $\partial S_k =
C_k$. If the deformation is such that each $C_k$ spans essentially the same
field region (for example, a tightly packed set of similar loops in a nearly
uniform field), then to a very good approximation
\begin{equation}
\Phi_k(t) \approx \Phi_0(t)
\qquad \text{for all } k=1,\dots,N.
\end{equation}
Substituting into \eqref{eq:emf-sum-subloops} yields
\begin{equation}
\mathcal{E}(t)
= \oint_C \mathbf{E}\cdot d\boldsymbol{\ell}
\approx -\sum_{k=1}^N \frac{d\Phi_0(t)}{dt}
= -\frac{d}{dt}\bigl(N\Phi_0(t)\bigr).
\end{equation}
Comparing with the defining relation \eqref{eq:emf-flux-linkage-def},
\[
\mathcal{E}(t) = -\frac{d\lambda(t)}{dt},
\]
we see that the natural choice for the flux linkage associated with the
deformed path $C$ is
\begin{equation}
\lambda(t) \approx N\,\Phi_0(t).
\label{eq:lambda-N-phi}
\end{equation}

This reasoning shows that what matters for the induced EMF is not the
geometric area of the original loop alone, but how many times the conducting
path \emph{links} the magnetic flux. A topological deformation that causes the
conductor to pass through the same field region $N$ times effectively multiplies
the flux linkage by $N$, even though the circuit is still a single closed path.

In the abstract notation of \eqref{eq:lambda-functional}, such a deformation
corresponds to replacing the operator $\mathcal{L}$ by a new operator
$\mathcal{L}_N$ that counts $N$ linkages of the same flux tube. Linearity of
$\mathcal{L}$ then implies
\begin{equation}
\lambda_N(t) = \mathcal{L}_N[\mathbf{B}(\cdot,t)]
\approx N\,\mathcal{L}_1[\mathbf{B}(\cdot,t)]
= N\,\lambda_1(t),
\end{equation}
in agreement with \eqref{eq:lambda-N-phi}. Later, when specific stator
constructions are introduced, these deformations will correspond to familiar
multi-pass conductor geometries, but the underlying principle is purely
topological: flux linkage measures how many times the circuit threads the
magnetic field, and this number can be increased by deforming the path so that
it passes through the flux region multiple times.

\section{Rotor Dynamics, Back-EMF, and Maximum Speed}

The previous sections characterized the air-gap field and showed how an ideal
rotor--stator pair of spatial order $n$ produces electromagnetic torque,
with a maximum torque
\begin{equation}
T_\mathrm{max}
= \frac{\pi R_r^2 L}{\mu_0}\, a b,
\end{equation}
when the rotor and stator field harmonics are perfectly aligned. We now turn to
the dynamics of the rotor under this torque and explain why the motor does not
accelerate indefinitely. The key additional ingredient is the induced voltage
(back-EMF) that arises from electromagnetic induction.

\subsection{Mechanical dynamics of the rotor}

The rotor can be modeled as a rigid body with moment of inertia $J$ about the
rotation axis. Let $\omega_m$ denote the mechanical angular velocity of the
rotor about the $z$-axis, and let $T_\mathrm{e}$ be the electromagnetic torque
exerted on the rotor by the air-gap field. The rotational equation of motion is
\begin{equation}
J \frac{d\omega_m}{dt}
= T_\mathrm{e} - T_\mathrm{L} - T_\mathrm{loss},
\label{eq:rotor-dynamics}
\end{equation}
where $T_\mathrm{L}$ represents the external load torque and $T_\mathrm{loss}$
represents frictional and other parasitic torques.

In the idealized limit of zero load and zero loss,
$T_\mathrm{L} = T_\mathrm{loss} = 0$, \eqref{eq:rotor-dynamics} reduces to
\begin{equation}
J \frac{d\omega_m}{dt} = T_\mathrm{e}.
\end{equation}
If $T_\mathrm{e}$ were a strictly constant positive torque, the solution would
be
\[
\omega_m(t) = \omega_m(0) + \frac{T_\mathrm{e}}{J} t,
\]
so the rotor speed would increase without bound. In reality, however, the
electromagnetic torque depends on the stator current, and the stator current is
limited by the applied voltage and by the induced voltage (back-EMF) generated
by the rotating magnetic field. As the rotor speeds up, the back-EMF grows and
reduces the net current and torque, so that the acceleration eventually ceases.

To make this precise we must relate the rotor motion to the induced voltage in
the stator windings.

\subsection{Electromagnetic induction and back-EMF}

Back-EMF is a direct consequence of Faraday's law of induction. For any closed
conducting loop $C$ bounding a surface $S$ with unit normal $\hat{\mathbf{n}}$,
Faraday's law states that the electromotive force (EMF) around $C$ is
\begin{equation}
\mathcal{E}
= \oint_C \mathbf{E}\cdot d\boldsymbol{\ell}
= -\frac{d}{dt}\int_S \mathbf{B}\cdot \hat{\mathbf{n}}\, dS.
\label{eq:Faraday-loop}
\end{equation}
For a stator phase winding with $N$ series turns, the total flux linkage
$\lambda(t)$ is $N$ times the flux through a single turn, and the induced phase
voltage is
\begin{equation}
e(t) = -\frac{d\lambda(t)}{dt}.
\label{eq:e-dlambda-dt}
\end{equation}

In a permanent-magnet synchronous motor the rotor magnets produce a
time-varying magnetic field in the stator reference frame whenever the rotor
rotates. For an ideal rotor of spatial order $n$, the radial air-gap flux
density at a fixed radius $R$ can be written as a traveling wave
\begin{equation}
B_r^r(R,\theta,t)
= \widehat{B}_r \cos\bigl(n(\theta - \omega_m t)\bigr),
\label{eq:Br-rotor-traveling-wave}
\end{equation}
where $\omega_m$ is the mechanical angular velocity and $n$ is the number of
pole pairs. The corresponding electrical angular frequency of the field is
\begin{equation}
\omega_e = n\,\omega_m,
\label{eq:omega-e-omega-m}
\end{equation}
which is the familiar synchronous-speed relation.

Consider one stator phase winding whose spatial distribution can be described
by a winding function $w(\theta)$ giving the effective turns per unit angle as
a function of position around the air-gap. For a sinusoidally distributed phase
of order $n$ we idealize
\begin{equation}
w(\theta) = \widehat{w} \cos\bigl(n(\theta - \theta_s)\bigr),
\end{equation}
where $\theta_s$ locates the phase axis and $\widehat{w}$ is a constant
proportional to the number of series turns. The flux linkage of this phase with
the rotor field is obtained by integrating the normal flux density weighted by
$w(\theta)$ over the air-gap and along the stack length $L$:
\begin{equation}
\lambda(t)
= \int_0^{2\pi} w(\theta)\,B_r^r(R,\theta,t)\,R L\, d\theta.
\label{eq:lambda-wB}
\end{equation}
Substituting \eqref{eq:Br-rotor-traveling-wave} gives
\begin{equation}
\lambda(t)
= \widehat{B}_r \widehat{w} R L
  \int_0^{2\pi}
  \cos\bigl(n(\theta - \theta_s)\bigr)
  \cos\bigl(n(\theta - \omega_m t)\bigr)\, d\theta.
\end{equation}
Using the identity
\[
\cos\alpha\,\cos\beta
= \tfrac{1}{2}\bigl[\cos(\alpha-\beta) + \cos(\alpha+\beta)\bigr]
\]
with $\alpha = n(\theta - \theta_s)$ and $\beta = n(\theta - \omega_m t)$, we
obtain
\begin{align}
\lambda(t)
&= \frac{\widehat{B}_r \widehat{w} R L}{2}
   \int_0^{2\pi}
   \bigl[
   \cos\bigl(n(\omega_m t - \theta_s)\bigr)
 + \cos\bigl(2n\theta - n(\theta_s + \omega_m t)\bigr)
   \bigr] d\theta \nonumber\\
&= \pi \widehat{B}_r \widehat{w} R L
   \cos\bigl(n(\omega_m t - \theta_s)\bigr),
\end{align}
since the second cosine term integrates to zero over a full $2\pi$ period. Thus
the phase flux linkage is sinusoidal in time with angular frequency
$\omega_e = n\omega_m$:
\begin{equation}
\lambda(t)
= \widehat{\lambda}
  \cos\bigl(\omega_e t - \phi_\lambda\bigr),
\qquad
\widehat{\lambda} = \pi \widehat{B}_r \widehat{w} R L,
\quad
\phi_\lambda = n\theta_s.
\end{equation}
Applying \eqref{eq:e-dlambda-dt} yields the induced phase voltage (back-EMF)
\begin{equation}
e(t)
= -\frac{d\lambda}{dt}
= \omega_e \widehat{\lambda}
  \sin\bigl(\omega_e t - \phi_\lambda\bigr).
\end{equation}
The peak back-EMF amplitude is therefore
\begin{equation}
\widehat{E}
= \omega_e \widehat{\lambda}
= n\,\omega_m\,\widehat{\lambda}.
\label{eq:Ehat-omega}
\end{equation}
This shows explicitly that, for a given rotor magnetization and winding
distribution, the back-EMF amplitude is proportional to the rotor speed:
\begin{equation}
\widehat{E} = k_e\,\omega_m,
\qquad
k_e = n\,\widehat{\lambda},
\label{eq:ke-definition}
\end{equation}
where $k_e$ is the back-EMF constant of the machine in SI units.

\subsection{Torque production, current, and the torque constant}

In the idealized air-gap model the torque expression
\eqref{eq:T-harmonic-amplitude-phase} shows that, for a single dominant spatial
harmonic of order $n$,
\[
T
\approx \frac{\pi R^2 L}{\mu_0}
       B_{r,n} B_{\theta,n}
       \cos\bigl(\phi_{r,n} - \phi_{\theta,n}\bigr).
\]
For an ideal rotor--stator pair with aligned fields we have
$\phi_{r,n} = \phi_{\theta,n}$ and
\[
T
= \frac{\pi R^2 L}{\mu_0} B_{r,n} B_{\theta,n}.
\]
The tangential harmonic amplitude $B_{\theta,n}$ is produced by the stator
current. For a sinusoidal, balanced stator winding, the fundamental air-gap
field amplitude is proportional to the peak stator phase current, so we may
write
\begin{equation}
B_{\theta,n} = \alpha\,\widehat{I},
\label{eq:Btheta-alpha-I}
\end{equation}
for some constant $\alpha$ determined by the winding and air-gap geometry.
Substituting \eqref{eq:Btheta-alpha-I} and identifying $B_{r,n}$ with the rotor
field amplitude gives
\begin{equation}
T
= \frac{\pi R^2 L}{\mu_0} B_{r,n} \alpha\,\widehat{I}
\equiv k_t\,\widehat{I},
\label{eq:T-ktI}
\end{equation}
where
\begin{equation}
k_t = \frac{\pi R^2 L}{\mu_0} B_{r,n} \alpha
\end{equation}
is the torque constant of the machine. In the sinusoidal, balanced case with
appropriate choice of units, the back-EMF constant $k_e$ and torque constant
$k_t$ are equal ($k_e = k_t$), reflecting conservation of energy: the
electromagnetic power $T \omega_m$ equals the air-gap power associated with the
product of back-EMF and stator current.

\subsection{Voltage limitation and maximum speed}

We now combine the back-EMF relation \eqref{eq:ke-definition} with the torque
relation \eqref{eq:T-ktI} to show why the motor does not accelerate forever
under a finite applied voltage.

Consider a single stator phase in steady-state sinusoidal operation with peak
applied phase voltage $\widehat{V}$ and phase resistance $R_s$. Neglecting
leakage inductance for the moment, the instantaneous phase voltage satisfies
\begin{equation}
v(t) = R_s i(t) + e(t),
\end{equation}
so at the level of fundamental amplitudes we have
\begin{equation}
\widehat{V} = R_s \widehat{I} + \widehat{E}.
\label{eq:V-I-E}
\end{equation}
Using \eqref{eq:ke-definition} we can write
\begin{equation}
\widehat{I}
= \frac{\widehat{V} - \widehat{E}}{R_s}
= \frac{\widehat{V} - k_e \omega_m}{R_s}.
\label{eq:I-vs-speed}
\end{equation}
Substituting into \eqref{eq:T-ktI} yields the torque as a function of rotor
speed in this idealized model:
\begin{equation}
T(\omega_m)
= k_t \widehat{I}
= \frac{k_t}{R_s}\bigl(\widehat{V} - k_e \omega_m\bigr).
\label{eq:torque-speed-linear}
\end{equation}
This expression shows that, for a given supply voltage $\widehat{V}$ and
machine constants $k_e,k_t,R_s$, the torque decreases linearly with rotor
speed. The electromagnetic torque becomes zero when
\[
\widehat{V} - k_e \omega_m = 0,
\]
i.e.\ when
\begin{equation}
\omega_m = \omega_\mathrm{no\mbox{-}load}
= \frac{\widehat{V}}{k_e}.
\label{eq:omega-noload}
\end{equation}
In the ideal case with no load torque and no losses, \eqref{eq:rotor-dynamics}
implies that the rotor accelerates from rest until it approaches the speed
$\omega_\mathrm{no\mbox{-}load}$, at which point the current and torque vanish
and the acceleration ceases. The motor cannot accelerate beyond this speed
under the given supply voltage because any further increase in speed would
require a current that violates \eqref{eq:V-I-E}.

In the more realistic case with a nonzero load torque $T_\mathrm{L}$ and
losses $T_\mathrm{loss}$, the steady-state speed is determined by the balance
condition
\[
T(\omega_m) = T_\mathrm{L} + T_\mathrm{loss},
\]
with $T(\omega_m)$ given by \eqref{eq:torque-speed-linear}. This yields a
speed lower than $\omega_\mathrm{no\mbox{-}load}$, but the qualitative picture
remains: back-EMF grows with speed, reduces the stator current and torque, and
thereby sets a finite maximum speed for a given applied voltage in the ideal
machine.
The previous section characterized the air-gap field and showed how an ideal
rotor--stator pair of spatial order $n$ produces electromagnetic torque, with a
maximum torque
\begin{equation}
  T_\mathrm{max} = \frac{\pi R^2 L}{\mu_0} B^r_r B^s_\theta,
\end{equation}
when the rotor and stator field harmonics are perfeclty aligned. Now the dynamics
of the rotor under this torque will be shown and why the motor does not accelerate
indefinitely will be explained.  The key additional ingredient is the induced
voltage (back-EMF) that arises as a result of Faraday's Law of electromagnetic
induction.

\subsection{Mechanical Dynamics of the Rotor}
The rotor can be modeled as a rigid body with moment of inertia $J$ about the
rotation axis.  Let $\omega_m$ denote the mechanical angular velocity of the 
rotor about the $z$-axis, and let $T_\mathrm{e}$ be the electromagnetic torque 
exterted on the rotor by the air-gap field.  The rotational equation of motion
is
\begin{equation}
J\frac{d\omega_m}{dt} = T_\mathrm{e} - T_\mathrm{L} - T_\mathrm{loss},
\label{eq:rotor-dynamics}
\end{equation}

\subsection{From standing waves to rotating fields: the need for multiple phases}

The stator conductors are rigidly attached to the stator and therefore fixed in
the stator reference frame. Their spatial distribution around the air-gap is
purely a function of the angular coordinate $\theta$ and does not move in time.
When currents flow in these conductors, the resulting current density and
current sheet distributions inherit this property: at any given time the
current pattern is some fixed function of $\theta$ multiplied by a time-varying
amplitude. As a consequence, a single sinusoidally distributed stator circuit
fed by a single sinusoidal current can produce at best a \emph{standing}
sinusoidal field pattern in the air-gap, not a rotating one. To obtain a
traveling wave of magnetic field, more than one stator circuit (phase) is
required.

\paragraph{Single-phase excitation: a standing wave.}

Let $w(\theta)$ denote the (time-independent) spatial distribution of one
stator circuit, expressed as an effective turns-per-unit-angle function. For a
sinusoidally distributed circuit of spatial order $n$ we idealize
\begin{equation}
w(\theta) = \widehat{w}\,\cos\bigl(n(\theta - \theta_s)\bigr),
\label{eq:w-single-phase}
\end{equation}
where $\theta_s$ fixes the location of the phase axis and $\widehat{w}$ is a
constant. If this circuit carries a sinusoidal current
\begin{equation}
i(t) = \widehat{I}\cos(\omega t),
\label{eq:i-single-phase}
\end{equation}
then, under the current-sheet approximation, the resulting surface current
density on the stator inner surface may be written as
\begin{equation}
K_\mathrm{f}(\theta,t)
= k\, i(t)\, w(\theta)
= k\,\widehat{I}\widehat{w}
  \cos(\omega t)\cos\bigl(n(\theta - \theta_s)\bigr),
\label{eq:K-single-phase}
\end{equation}
where $k$ is a proportionality constant depending on slot geometry. Using the
trigonometric identity
\[
\cos A\,\cos B
= \tfrac{1}{2}\bigl[\cos(A+B) + \cos(A-B)\bigr],
\]
and setting $A = \omega t$, $B = n(\theta - \theta_s)$, we obtain
\begin{align}
K_\mathrm{f}(\theta,t)
&= \frac{k\,\widehat{I}\widehat{w}}{2}
   \Bigl[
   \cos\bigl(\omega t + n(\theta - \theta_s)\bigr)
 + \cos\bigl(\omega t - n(\theta - \theta_s)\bigr)
   \Bigr].
\label{eq:K-standing}
\end{align}
Each cosine term can be interpreted as a traveling wave. For example, the
locus of constant phase for $\cos(\omega t - n(\theta - \theta_s))$ satisfies
$\omega t - n(\theta - \theta_s) = \text{const}$, i.e.
\[
\theta - \theta_s = \frac{\omega}{n}\,t + \text{const},
\]
which corresponds to a wave pattern rotating in the positive $\theta$-direction
with angular velocity $\omega/n$. The other term,
$\cos(\omega t + n(\theta - \theta_s))$, represents a wave of equal amplitude
rotating in the opposite direction at the same angular speed. Thus the current
sheet produced by a single stationary sinusoidal winding and a single
sinusoidal current is the superposition of two counter-propagating traveling
waves of equal amplitude.

Because the governing magnetostatic equations in the air-gap are linear, the
tangential component of the stator-generated magnetic field $B_\theta^s$ at the
air-gap boundary is proportional to $K_\mathrm{f}$,
\[
B_\theta^s(R_s,\theta,t)
\propto K_\mathrm{f}(\theta,t),
\]
and therefore inherits the same structure: it is a standing wave formed by the
superposition of two equal-amplitude waves rotating in opposite directions.
Although $B_\theta^s$ is sinusoidal in time at any fixed $\theta$, and
sinusoidal in space at any fixed $t$, the field pattern as a whole does not
rotate steadily in one direction. Consequently, a single-phase stator cannot
produce a continuously rotating air-gap field and can sustain at most a
time-varying torque with large oscillations and zero average starting torque.

\paragraph{Multiple phases and traveling waves.}

To obtain a unidirectional traveling wave of magnetic field, we must design the
stator so that the counter-rotating components of the field cancel, leaving
only one traveling wave. This can be achieved by using several spatially
displaced stator circuits (phases) carrying time-shifted currents. Each phase
alone produces a standing wave, but their superposition can be arranged so
that the backward-traveling waves cancel while the forward-traveling waves
reinforce.

Consider three identical sinusoidally distributed stator circuits of spatial
order $n$, with spatial phase shifts of $120^\circ$ (i.e.\ $2\pi/3$ radians)
between them. Let their spatial distributions be
\begin{align}
w_a(\theta) &= \widehat{w}\,\cos(n\theta), \\
w_b(\theta) &= \widehat{w}\,\cos\bigl(n\theta - \tfrac{2\pi}{3}\bigr), \\
w_c(\theta) &= \widehat{w}\,\cos\bigl(n\theta + \tfrac{2\pi}{3}\bigr).
\end{align}
Assume that these three circuits carry balanced three-phase currents
\begin{align}
i_a(t) &= \widehat{I}\cos(\omega t), \\
i_b(t) &= \widehat{I}\cos\bigl(\omega t - \tfrac{2\pi}{3}\bigr), \\
i_c(t) &= \widehat{I}\cos\bigl(\omega t + \tfrac{2\pi}{3}\bigr).
\end{align}
The resulting surface current density on the stator inner surface is
\begin{equation}
K_\mathrm{f}(\theta,t)
= k\Bigl[
    i_a(t) w_a(\theta)
  + i_b(t) w_b(\theta)
  + i_c(t) w_c(\theta)
  \Bigr].
\label{eq:K-3phase-sum}
\end{equation}
Substituting the explicit expressions gives
\begin{align}
K_\mathrm{f}(\theta,t)
&= k\,\widehat{I}\widehat{w}
   \Bigl[
   \cos(\omega t)\cos(n\theta)
 + \cos\bigl(\omega t - \tfrac{2\pi}{3}\bigr)
   \cos\bigl(n\theta - \tfrac{2\pi}{3}\bigr) \nonumber\\
&\hspace{7em}
 + \cos\bigl(\omega t + \tfrac{2\pi}{3}\bigr)
   \cos\bigl(n\theta + \tfrac{2\pi}{3}\bigr)
   \Bigr].
\end{align}
Applying the product-to-sum identity to each term, and simplifying, one finds
(after straightforward but slightly lengthy algebra) that all backward-rotating
components cancel and the sum reduces to a single forward-traveling wave:
\begin{equation}
K_\mathrm{f}(\theta,t)
= \frac{3}{2}\,k\,\widehat{I}\widehat{w}
  \cos\bigl(n\theta - \omega t\bigr).
\label{eq:K-traveling-3phase}
\end{equation}
A direct derivation uses
\[
\cos\alpha\,\cos\beta
= \tfrac{1}{2}\bigl[\cos(\alpha-\beta) + \cos(\alpha+\beta)\bigr]
\]
and the identity
\[
\cos x + \cos\Bigl(x + \tfrac{2\pi}{3}\Bigr)
       + \cos\Bigl(x + \tfrac{4\pi}{3}\Bigr) = 0,
\]
which ensures that the backward-traveling terms vanish in the three-phase sum.

Equation \eqref{eq:K-traveling-3phase} has the form of a single traveling wave
of constant amplitude: the current sheet is proportional to
\[
\cos\bigl(n\theta - \omega t\bigr),
\]
which represents a wave of spatial order $n$ rotating in the positive
$\theta$-direction with angular velocity $\omega/n$ and constant amplitude
$\tfrac{3}{2}k\widehat{I}\widehat{w}$. Because the air-gap field
$\mathbf{B}^s$ is linearly related to $K_\mathrm{f}$ through the magnetostatic
field equations, the fundamental component of the stator-generated magnetic
field is likewise a rotating sinusoidal field of constant amplitude. This is
the \emph{rotating magnetic field} that interacts with the rotor field to
produce a steady electromagnetic torque.

In summary, the stationary nature of the stator conductors means that a single
sinusoidally distributed stator circuit excited by a sinusoidal current
produces a standing wave in the air-gap, composed of two counter-propagating
traveling waves of equal amplitude. By using multiple spatially displaced
circuits with time-shifted currents---in particular, a balanced three-phase
system with $120^\circ$ separation in both space and time---the backward
components cancel and a single traveling wave of constant amplitude is
obtained. This rotating stator field is essential for continuous torque
production in a synchronous permanent-magnet motor.

\subsection{From standing waves to rotating fields: the need for multiple phases}

The stator conductors are rigidly attached to the stator and therefore fixed in
the stator reference frame. Their spatial distribution around the air-gap is
purely a function of the angular coordinate $\theta$ and does not move in time.
When currents flow in these conductors, the resulting current density and
current sheet distributions inherit this property: at any given time the
current pattern is some fixed function of $\theta$ multiplied by a time-varying
amplitude. As a consequence, a single sinusoidally distributed stator circuit
fed by a single sinusoidal current can produce at best a \emph{standing}
sinusoidal field pattern in the air-gap, not a rotating one. To obtain a
traveling wave of magnetic field, more than one stator circuit (phase) is
required.

\paragraph{Single-phase excitation: a standing wave.}

Let $w(\theta)$ denote the (time-independent) spatial distribution of one
stator circuit, expressed as an effective turns-per-unit-angle function. For a
sinusoidally distributed circuit of spatial order $n$ we idealize
\begin{equation}
w(\theta) = \widehat{w}\,\cos\bigl(n(\theta - \theta_s)\bigr),
\label{eq:w-single-phase}
\end{equation}
where $\theta_s$ fixes the location of the phase axis and $\widehat{w}$ is a
constant. If this circuit carries a sinusoidal current
\begin{equation}
i(t) = \widehat{I}\cos(\omega t),
\label{eq:i-single-phase}
\end{equation}
then, under the current-sheet approximation, the resulting surface current
density on the stator inner surface may be written as
\begin{equation}
K_\mathrm{f}(\theta,t)
= k\, i(t)\, w(\theta)
= k\,\widehat{I}\widehat{w}
  \cos(\omega t)\cos\bigl(n(\theta - \theta_s)\bigr),
\label{eq:K-single-phase}
\end{equation}
where $k$ is a proportionality constant depending on slot geometry. Using the
trigonometric identity
\[
\cos A\,\cos B
= \tfrac{1}{2}\bigl[\cos(A+B) + \cos(A-B)\bigr],
\]
and setting $A = \omega t$, $B = n(\theta - \theta_s)$, we obtain
\begin{align}
K_\mathrm{f}(\theta,t)
&= \frac{k\,\widehat{I}\widehat{w}}{2}
   \Bigl[
   \cos\bigl(\omega t + n(\theta - \theta_s)\bigr)
 + \cos\bigl(\omega t - n(\theta - \theta_s)\bigr)
   \Bigr].
\label{eq:K-standing}
\end{align}
Each cosine term can be interpreted as a traveling wave. For example, the
locus of constant phase for $\cos(\omega t - n(\theta - \theta_s))$ satisfies
$\omega t - n(\theta - \theta_s) = \text{const}$, i.e.
\[
\theta - \theta_s = \frac{\omega}{n}\,t + \text{const},
\]
which corresponds to a wave pattern rotating in the positive $\theta$-direction
with angular velocity $\omega/n$. The other term,
$\cos(\omega t + n(\theta - \theta_s))$, represents a wave of equal amplitude
rotating in the opposite direction at the same angular speed. Thus the current
sheet produced by a single stationary sinusoidal winding and a single
sinusoidal current is the superposition of two counter-propagating traveling
waves of equal amplitude.

Because the governing magnetostatic equations in the air-gap are linear, the
tangential component of the stator-generated magnetic field $B_\theta^s$ at the
air-gap boundary is proportional to $K_\mathrm{f}$,
\[
B_\theta^s(R_s,\theta,t)
\propto K_\mathrm{f}(\theta,t),
\]
and therefore inherits the same structure: it is a standing wave formed by the
superposition of two equal-amplitude waves rotating in opposite directions.
Although $B_\theta^s$ is sinusoidal in time at any fixed $\theta$, and
sinusoidal in space at any fixed $t$, the field pattern as a whole does not
rotate steadily in one direction. Consequently, a single-phase stator cannot
produce a continuously rotating air-gap field and can sustain at most a
time-varying torque with large oscillations and zero average starting torque.

\paragraph{Multiple phases and traveling waves.}

To obtain a unidirectional traveling wave of magnetic field, we must design the
stator so that the counter-rotating components of the field cancel, leaving
only one traveling wave. This can be achieved by using several spatially
displaced stator circuits (phases) carrying time-shifted currents. Each phase
alone produces a standing wave, but their superposition can be arranged so
that the backward-traveling waves cancel while the forward-traveling waves
reinforce.

Consider three identical sinusoidally distributed stator circuits of spatial
order $n$, with spatial phase shifts of $120^\circ$ (i.e.\ $2\pi/3$ radians)
between them. Let their spatial distributions be
\begin{align}
w_a(\theta) &= \widehat{w}\,\cos(n\theta), \\
w_b(\theta) &= \widehat{w}\,\cos\bigl(n\theta - \tfrac{2\pi}{3}\bigr), \\
w_c(\theta) &= \widehat{w}\,\cos\bigl(n\theta + \tfrac{2\pi}{3}\bigr).
\end{align}
Assume that these three circuits carry balanced three-phase currents
\begin{align}
i_a(t) &= \widehat{I}\cos(\omega t), \\
i_b(t) &= \widehat{I}\cos\bigl(\omega t - \tfrac{2\pi}{3}\bigr), \\
i_c(t) &= \widehat{I}\cos\bigl(\omega t + \tfrac{2\pi}{3}\bigr).
\end{align}
The resulting surface current density on the stator inner surface is
\begin{equation}
K_\mathrm{f}(\theta,t)
= k\Bigl[
    i_a(t) w_a(\theta)
  + i_b(t) w_b(\theta)
  + i_c(t) w_c(\theta)
  \Bigr].
\label{eq:K-3phase-sum}
\end{equation}
Substituting the explicit expressions gives
\begin{align}
K_\mathrm{f}(\theta,t)
&= k\,\widehat{I}\widehat{w}
   \Bigl[
   \cos(\omega t)\cos(n\theta)
 + \cos\bigl(\omega t - \tfrac{2\pi}{3}\bigr)
   \cos\bigl(n\theta - \tfrac{2\pi}{3}\bigr) \nonumber\\
&\hspace{7em}
 + \cos\bigl(\omega t + \tfrac{2\pi}{3}\bigr)
   \cos\bigl(n\theta + \tfrac{2\pi}{3}\bigr)
   \Bigr].
\end{align}
Applying the product-to-sum identity to each term, and simplifying, one finds
(after straightforward but slightly lengthy algebra) that all backward-rotating
components cancel and the sum reduces to a single forward-traveling wave:
\begin{equation}
K_\mathrm{f}(\theta,t)
= \frac{3}{2}\,k\,\widehat{I}\widehat{w}
  \cos\bigl(n\theta - \omega t\bigr).
\label{eq:K-traveling-3phase}
\end{equation}
A direct derivation uses
\[
\cos\alpha\,\cos\beta
= \tfrac{1}{2}\bigl[\cos(\alpha-\beta) + \cos(\alpha+\beta)\bigr]
\]
and the identity
\[
\cos x + \cos\Bigl(x + \tfrac{2\pi}{3}\Bigr)
       + \cos\Bigl(x + \tfrac{4\pi}{3}\Bigr) = 0,
\]
which ensures that the backward-traveling terms vanish in the three-phase sum.

Equation \eqref{eq:K-traveling-3phase} has the form of a single traveling wave
of constant amplitude: the current sheet is proportional to
\[
\cos\bigl(n\theta - \omega t\bigr),
\]
which represents a wave of spatial order $n$ rotating in the positive
$\theta$-direction with angular velocity $\omega/n$ and constant amplitude
$\tfrac{3}{2}k\widehat{I}\widehat{w}$. Because the air-gap field
$\mathbf{B}^s$ is linearly related to $K_\mathrm{f}$ through the magnetostatic
field equations, the fundamental component of the stator-generated magnetic
field is likewise a rotating sinusoidal field of constant amplitude. This is
the \emph{rotating magnetic field} that interacts with the rotor field to
produce a steady electromagnetic torque.

In summary, the stationary nature of the stator conductors means that a single
sinusoidally distributed stator circuit excited by a sinusoidal current
produces a standing wave in the air-gap, composed of two counter-propagating
traveling waves of equal amplitude. By using multiple spatially displaced
circuits with time-shifted currents---in particular, a balanced three-phase
system with $120^\circ$ separation in both space and time---the backward
components cancel and a single traveling wave of constant amplitude is
obtained. This rotating stator field is essential for continuous torque
production in a synchronous permanent-magnet motor.

\begin{figure}[h!]
    \centering
    \includegraphics[width=0.5\textwidth]{plots/img/rotor_analytic.png}
    \caption{The magnetization of an ideal permanent magnet rotor.}
    \label{fig:rotor}
\end{figure}

\begin{figure}[h!]
    \centering
    \includegraphics[width=\textwidth]{plots/img/rotor_analytic_magnetization.png}
    \caption{The magnetic field outside an ideal permanent magnet rotor.}
    \label{fig:rotor_magnetization}
\end{figure}

\begin{figure}[h!]
    \centering
    \includegraphics[width=\textwidth]{plots/img/rotor_and_stator_air_gap.png}
    \caption{The magnetic field of an ideal stator and rotor decomposing the
    field in term of the component contributions from the stator and rotor.
    The stator field is entirely tangential, while the rotor field is entirely
    radial at the center of the magnetic poles.}
    \label{fig:rotor-stator-air-gap}
\end{figure}

\begin{figure}[h!]
    \centering
    \includegraphics[width=\textwidth]{plots/img/rotor_and_stator_torque_density.png}
    \caption{The tangential shear component of the Maxwell stress tensor in the
    air-gap. This gives the torque density at each point. The torque denisty is
    highest along the surface of the rotor.}
    \label{fig:rotor-stator-torque-density}
\end{figure}

\end{document}