\documentclass[11pt]{article}

% Encoding and language
\usepackage[utf8]{inputenc}
\usepackage[T1]{fontenc}
\usepackage[english]{babel}

% Math and symbols
\usepackage{amsmath, 
            amssymb,
            graphicx,
            subcaption}
\usepackage[europeanresistors,americaninductors]{circuitikz}

% Better spacing and layout
\usepackage{microtype}
\usepackage{geometry}
\geometry{margin=1in}

% Hyperlinks
\usepackage[colorlinks=true, linkcolor=blue, urlcolor=blue, citecolor=blue]{hyperref}

\title{Permanent Magnet Motor Control}
\author{Anthony J. Steel}
\date{\today}

\begin{document}

\maketitle
\section{Introduction}

The fundamental electromagnetic unit of a motor is the interaction of two fields
in a region called the \emph{air-gap}. The goal of motor control is to manipulate
the fields in the air-gap to control relevant physical quantities such as
torque, speed, position, energy, and power. In a permanent-magnet motor one of
these fields is produced by permanent magnets and the other is produced by
currents. The mechanism that contains the permanent magnets is called the
\emph{rotor} because it is free to rotate, while the one that carries the currents
is called the \emph{stator} because it is generally stationary relative to the
labratory.

\section{Preliminaries}

\subsection{Maxwell's Equations}
Begin with the macroscopic Maxwell equations in SI units,

\begin{align}
\nabla \cdot \mathbf{B} &= 0,\\
\nabla \times \mathbf{E} &= -\frac{\partial \mathbf{B}}{\partial t},\\
\nabla \cdot \mathbf{D} &= \rho_\mathrm{f},\\
\nabla \times \mathbf{H} &= \mathbf{J}_\mathrm{f} + \frac{\partial\mathbf{D}}{\partial t},
\end{align}
where $\rho_\mathrm{f}$ and $\mathbf{J}_\mathrm{f}$ denote the free charge and
free current densities, and
\[
\mathbf{B} = \mu_0(\mathbf{H} + \mathbf{M}),\qquad
\mathbf{D} = \varepsilon_0 \mathbf{E} + \mathbf{P}
\]
are the magnetic flux density and electric displacement expressed in terms of
the magnetic and electric polarizations $\mathbf{M}$ and $\mathbf{P}$.

The air-gap region $V_g$ lies between the rotor and stator surfaces as shown in
Figure \ref{fig:air-gap}.  By construction, it has no current-carrying conductors
and no magnetic material, so in $V_g$
\[
\mathbf{J}_\mathrm{f} = \mathbf{0},\qquad \mathbf{M} = \mathbf{0}.
\]
Furthermore, as the name suggests, the gap is filled with air which is approximated
as linear, homogenous and isotropic with
\[
\mathbf{B} = \mu_0 \mathbf{H}, \qquad \mathbf{D} = \varepsilon_0 \mathbf{E}.
\]

\subsection{Magnetoquasistatic Approximation}



\begin{figure}[h!]
    \centering
    \includegraphics[width=\textwidth]{plots/img/air_gap.png}
    \caption{The air-gap of a motor. The region of space where the stator and r
    otor fields interact to produce torque on the surface of the rotor. No
    fields are shown.}
    \label{fig:air-gap}
\end{figure}


The air-gap contains no free currents, so Maxwell's equations reduce to the
magnetostatic form called Laplace's equation:
\[
\nabla^2\Phi = 0,
\]
where $\Phi$ is the \emph{scalar magnetic potential}. The magnetic fields are
recovered from the scalar potential by $\mathbf{B} = -\mu_0 \nabla \Phi$. 
The physically admissible general solution to Laplace's equation in the air-gap
using 2D cylindrical coordinates is a linear superposition of Fourier modes with 
radial decay factors,
\[
\Phi(r,\theta) =
\sum_{n=1}^{\infty} A_n r^{-n}\cos(n\theta) + B_n r^{-n}\sin(n\theta).
\]
This representation of the solution is convenient when one wants to utilize the
orthogonality of the Fourier modes to simplify algebra, as will be done later, 
however it hides the phase relations between modes in the expansion 
coefficients. The equivalent representation
\[
\Phi(r,\theta) = 
\sum_{n=1}^\infty C_n r^{-n} \cos(n\phi - \psi_n)
\]
with $C_n = \sqrt{A_n^2 + B_n^2}$ and $\phi_n = \arctan(B_n, A_n)$ by using
suitable trignometric identities. Any field that can exist in the air-gap will
be represented by this solution with suitable coefficients and 
boundary-conditions.

In order for the magnetic fields generated in the air-gap to produce rotation,
they must exert a torque on the rotor. To understand this, one must describe
how electromagnetic fields apply forces to material bodies. Electromagnetic
forces arise from exchanges of momentum between the electromagnetic field and
matter. This momentum flux is described by the \emph{Maxwell stress tensor}, which
encodes the pressure and shear stresses that the fields exert on surfaces.

In cylindrical coordinates in the air-gap the Maxwell stress tensor $\mathbf{T}$
takes the form
\[
T_{ij} =
\frac{1}{\mu_0}\Big(B_i B_j - \tfrac{1}{2}\delta_{ij} B^2\Big),
\]
where the indices $i,j$ run over $r$ and $\theta$. These indices have geometric
meaning: $j$ specifies the direction of the surface normal on which the field
acts, while $i$ specifies the direction of the force that the field applies on 
that surface. In cylndrical coordinates, pressure corresponds to a force
exerted in the radial $r$ direction, while shear is a force exerted in
the tangential $\theta$ direction.

In a permanent-magnet motor, the surface that the electromagnetic forces are 
applied is the surface of the rotor, whose unit normal is $\hat{r}$, and the component of the force that
produces torque lies in the tangential $\hat{\theta}$ direction. Therefore the
\emph{only} component of the Maxwell stress tensor that produces torque on the rotor
is the $T_{\theta r}$ term, given by:
\[
T_{\theta r} = \frac{1}{\mu_0} B_r B_\theta.
\]
For a moment, let the radial and tangential fields have contributions
from both the rotor and stator fields, that is
\[
B_r = B_r^s + B_r^r,\qquad B_\theta = B_\theta^s + B_\theta^r
\]
where the superscript $r$ and $s$ denote the rotor and stator respectively.
Then the torque producing component of the Maxwell stress tensor is
\[
T_{\theta r} = \frac{1}{\mu_0} \Big(B_r^sB_\theta^s + B_r^sB_\theta^r + B_r^rB_\theta^s
+ B_r^rB_\theta^r)\Big)
\]
The two terms $B_r^sB_\theta^s$ and $B_r^rB_\theta^r$ are \emph{self-interaction}
terms which must produce zero torque.
To see this, if the rotor or stator is considered in isolation, its own field
cannot produce a net rigid-body torque on itself. The associated Maxwell stresses
are internal stresses that integrate to zero net torque. Therefore only the 
cross-terms can contribute to electromagnetic torque on the rotor.


The total electromagnetic torque is obtained by integrating this component
around the air-gap,
\[
T = \frac{R^2 L}{\mu_0}\int_0^{2\pi} B_r(\theta)\,B_\theta(\theta)\,d\theta.
\]
Both $B_r$ and $B_\theta$ admit Fourier expansions in $\theta$, as previously
shown:
\[
B_r = \sum_{n=1}^\infty a_n\cos(n\theta) + b_n\sin(n\theta),\qquad
B_\theta = \sum_{m=1}^\infty c_m \cos(m\theta) + b_m\sin(m\theta).
\]
Calculating the torque produces the rather long expression,
\[
\begin{gathered}
T = \frac{R^2L}{\mu_0}\int^{2\pi}_0\sum_{n=1}^\infty\sum_{m=1}^\infty 
a_nc_m\cos(n\theta)\cos(m\theta) + a_nb_m\cos(n\theta)\sin(m\theta) + \\
b_nc_m\sin(n\theta)\cos(m\theta) + b_nb_m\sin(n\theta)\sin(m\theta)\,d\theta.
\end{gathered}
\]
Luckily, the orthogonality relations for Fourier modes can be readily utilized,
namely
\[
\int_0^{2\pi}\cos(n\theta)\cos(m\theta)d\theta =  
\begin{cases}
0, & n\neq m\\
\pi, & n=m\neq0,
\end{cases}
\]
\[
\int_0^{2\pi}\sin(n\theta)\sin(m\theta) = 
\begin{cases}
0, & n\neq m\\
\pi, & n=m\neq0,
\end{cases}
\]
\[
\int_0^{2\pi}\cos(n\theta)\sin(m\theta)d\theta = 0\; \; \; \forall\, n,m,
\]
so that all cross-terms vanish and
\[
T = \frac{\pi R^2L}{\mu_0}\sum_{n=1}^\infty(a_nc_n + b_nd_n).
\]
To make the phase relations between modes explicit, use the transformations
discussed previously. Let $B_{r,n}=\sqrt{a_n^2 + b_n^2}\;$, $B_{\theta,n} =
\sqrt{c_n^2 + d_n^2}\;$, $\phi_{r,n} = \arctan(b_n, a_n)\;$, and $\phi_{\theta,n}
= \arctan(d_n, c_n)$. The torque becomes
\[
T = \frac{\pi R^2 L}{\mu_0}\sum_{n=1}^\infty B_{r,n}B_{\phi,n}
\cos(\phi_{\theta,n} - \phi_{r,n})
\]
This expression for torque is the first interesting result.
The net torque on the rotor is determined \emph{only} by harmonics of the same 
spatial order. Equivalently, the net torque is constant over the  surface of the 
rotor even though local torque density around the air-gap is not. The local spatial
variations in the torque density cancel out when the torque over the entire
air-gap is considered. 
\begin{figure}[h!]
    \centering
    \includegraphics[width=0.5\textwidth]{plots/img/rotor_analytic.png}
    \caption{The magnetization of an ideal permanent magnet rotor.}
    \label{fig:rotor}
\end{figure}

\begin{figure}[h!]
    \centering
    \includegraphics[width=\textwidth]{plots/img/rotor_analytic_magnetization.png}
    \caption{The magnetic field outside an ideal permanent magnet rotor.}
    \label{fig:rotor_magnetization}
\end{figure}

\begin{figure}[h!]
    \centering
    \includegraphics[width=\textwidth]{plots/img/rotor_and_stator_air_gap.png}
    \caption{The magnetic field of an ideal stator and rotor decomposing the
    field in term of the component contributions from the stator and rotor.
    The stator field is entirely tangential, while the rotor field is entirely
    radial at the center of the magnetic poles.}
    \label{fig:rotor-stator-air-gap}
\end{figure}

\begin{figure}[h!]
    \centering
    \includegraphics[width=\textwidth]{plots/img/rotor_and_stator_torque_density.png}
    \caption{The tangential shear component of the Maxwell stress tensor in the
    air-gap. This gives the torque density at each point. The torque denisty is
    highest along the surface of the rotor.}
    \label{fig:rotor-stator-torque-density}
\end{figure}

\end{document}