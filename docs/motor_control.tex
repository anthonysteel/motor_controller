\documentclass[11pt]{article}

% Encoding and language
\usepackage[utf8]{inputenc}
\usepackage[T1]{fontenc}
\usepackage[english]{babel}

% Math and symbols
\usepackage{amsmath, amssymb}

% Better spacing and layout
\usepackage{microtype}
\usepackage{geometry}
\geometry{margin=1in}

% Hyperlinks
\usepackage[colorlinks=true, linkcolor=blue, urlcolor=blue, citecolor=blue]{hyperref}

\title{Motor Control}
\author{Anthony J. Steel}
\date{\today}

\begin{document}

\maketitle

A 3-phase inverter contains 6 switches arranged in a 3 half-bridge topology.
Each half-bridge leg has a high and low side switch.  These switches must be
driven in a complimentary fashion to prevent grounding the source.  This 
complimentarity requirement reduces the $2^6$ possible switch states to $2^3$.
Of these 8 states, two correspond to zero or floating voltages when either
all the low-side/high-side switches are on respectively.

Now, each of the remaining 6 states will produce discrete line-to-neutral
voltages in the stator windings of the motor that they are connected to
These states are given by: 
\[
(s_a, s_b, s_c)\in \{0, 1\}^3
\]
It is
self-evident that it will be useful to produce voltages other than these 
6 discrete states in the stator windings through some method of
time-averaging.  The detailed discussion of this averaging is deffered for the 
moment.

What can be said mathematically about the set of physically valid line-to-neutral 
voltage triplets $(v_a, v_b, v_c)$? Firstly, it is naturally a vector space.  If all the 
low-side switches are on, every leg is grounded and there is therefore a zero vector
$(0, 0, 0)$. Given two voltage triplets $(v_a, v_b, v_c)$ and $(w_a, w_b, w_c)$
their sum $(v_a+w_a, v_b+w_b, v_c+w_c)$ is also a valid voltage, assuming that
the sum is less than the inverter source. An additive inverse exists because
the inverter can produce negative voltages.  Therefore it is natural to
associate each state of the inverter with a point in $\mathbb{R}^3$. 

Secondly, the set of possible points is confined by Kirchhoff's Voltage Law
to the plane $v_a + v_b + v_c = 0$. Therefore the voltages lie in:
\[
V = \{ (v_a, v_b, v_c) \in \mathbb{R}^3 \mid v_a + v_b + v_c = 0 \}
\]
Returning to the discrete states for a moment. These 6 discrete states form a 
hexagon in the voltage plane.
Since this plane is a two-dimensional subset of $\mathbb{R}^3$, it can be
completely described by two vectors lying in it.
\end{document}