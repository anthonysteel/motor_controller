\documentclass[11pt]{article}

% Encoding and language
\usepackage[utf8]{inputenc}
\usepackage[T1]{fontenc}
\usepackage[english]{babel}

% Math and symbols
\usepackage{amsmath, amssymb}

% Better spacing and layout
\usepackage{microtype}
\usepackage{geometry}
\geometry{margin=1in}

% Hyperlinks
\usepackage[colorlinks=true, linkcolor=blue, urlcolor=blue, citecolor=blue]{hyperref}

\title{Motor Control}
\author{Anthony J. Steel}
\date{\today}

\begin{document}

\maketitle

A 3-phase inverter contains 6 switches arranged in a 3 half-bridge topology.
Each half-bridge leg has a high and low side switch with a lead connected to a 
motor phase between the switches.  These switches must be driven in a 
complementary fashion to prevent grounding the source.  This  complementarity 
requirement reduces the $2^6$ possible switching states to $2^3$. Of these 8 
states, two correspond to zero or floating voltages when either all the 
low-side/high-side switches are on respectively.

Now, each of the remaining 6 states will produce discrete line-to-neutral
voltages in the stator windings of the motor that they are connected to.
These states are given by:  
\[
(s_a, s_b, s_c)\in \{0, 1\}^3
\]
What voltages can these switching states produce in the phases of the stator?
Let $V_\text{dc}$ be the DC bus voltage. Consider one leg of the inverter first
and let the voltage of this phase be $v_i$.  At any moment either the high-side
switch is on connecting the phase to $V_\text{dc}$ or the low-side switch is on
grounding it.  Thus the phase-node voltages are
\[
v_a, v_b, v_c \in \{0, V_\text{dc}\}.
\]
The set of all physically possible raw node voltages are the points
\[
\{0, V_\text{dc}\}^3 \subset \mathbb{R}^3
\]
which form the vertices of a cube. In fact, by suitable averaging, the inverter
can realize any point in the convex hull of these vertices.

However, it is important to note that there is a physical redunancy in this
representation because the current in the stator windings does not depend on raw
voltages. 
Faraday's Law states
\[
e_{ab} = \frac{d\lambda}{dt},
\]
and the electrical winding voltage is $v_{ab} = v_a - v_b$.
Therefore the current that flows in the stator depends on the voltage 
differences \emph{between} the phases, not their value relative to some ground.
This means that adding the same constant $k$, called a common-mode, to all three
leg voltages
\[
(v_a, v_b, v_c) \mapsto (v_a + k, v_b + k, v_c + k)
\]
leaves all phase-to-phase voltages $v_{ab}, v_{bc}, v_{ca}$ unchanged.
Hence, many points in the cube represent the same physical winding voltages
and currents.

To remove this redundancy, quotient out the common-mode by solving for it in terms
of the $v_a, v_b, v_c$ voltages.  If the stator is wound in a Y-connected with 
a floating neutral, Kirchoff's Current Law at the neutral node gives
\[
\frac{v_a - v_N}{Z_a} + \frac{v_b - v_N}{Z_b} + \frac{v_c - v_N}{Z_c} = 0
\]
For balanced phases $Z_a = Z_b = Z_c$, which simplifies to
\[
v_a - v_N + v_b + v_N + v_c + v_N = 0
\]
and therefore
\[
v_N = \frac{v_a + v_b + v_c}{3}.
\]
Identifying $v_N = -k$, referencing the voltages to neutral provides the 
constraint that elimnates the redundancy.

Geometrically, subtracting the neutral voltage is equivalent to projecting the
convex hull of the cube $[0, V_\text{dc}]^3$ onto the 2-dimensional plane
\[
v_{aN} + v_{bN} + v_{cN} = 0
\]
This is a linear projection along the direction of the vector $(1, 1, 1)$.

It is self-evident that it will be useful to produce voltages other than these 
6 discrete states in the stator windings through some method of
time-averaging.  The detailed discussion of this averaging is deffered for the 
moment.

What can be said mathematically about the set of physically valid line-to-neutral 
voltage triplets $(v_a, v_b, v_c)$? Firstly, it is naturally a vector space.  If all the 
low-side switches are on, every leg is grounded and there is therefore a zero vector
$(0, 0, 0)$. Given two voltage triplets $(v_a, v_b, v_c)$ and $(w_a, w_b, w_c)$
their sum $(v_a+w_a, v_b+w_b, v_c+w_c)$ is also a valid voltage, assuming that
the sum is less than the inverter source. An additive inverse exists because
the inverter can produce negative voltages.  Therefore it is natural to
associate each state of the inverter with a point in $\mathbb{R}^3$. 

Secondly, the set of possible points is confined by Kirchhoff's Voltage Law
to the plane $v_a + v_b + v_c = 0$. Therefore the voltages lie in:
\[
V = \{ (v_a, v_b, v_c) \in \mathbb{R}^3 \mid v_a + v_b + v_c = 0 \}
\]
Returning to the discrete states for a moment. These 6 discrete states form a 
hexagon in the voltage plane.
Since this plane is a two-dimensional subset of $\mathbb{R}^3$, it can be
completely described by two vectors lying in it.
\end{document}