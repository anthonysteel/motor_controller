\documentclass[11pt]{article}

% Encoding and language
\usepackage[utf8]{inputenc}
\usepackage[T1]{fontenc}
\usepackage[english]{babel}

% Math and symbols
\usepackage{amsmath, 
            amssymb,
            graphicx,
            subcaption}
\usepackage[europeanresistors,americaninductors]{circuitikz}

% Better spacing and layout
\usepackage{microtype}
\usepackage{geometry}
\geometry{margin=1in}

% Hyperlinks
\usepackage[colorlinks=true, linkcolor=blue, urlcolor=blue, citecolor=blue]{hyperref}

\title{Permanent-Magnet Motor Control}
\author{Anthony J. Steel}
\date{\today}

\begin{document}

\maketitle
\section{Introduction}

The fundamental electromagnetic unit of a motor is the interaction of two fields
in a region called the \emph{air-gap}. The goal of motor control is to manipulate
the fields in the air-gap to control relevant physical quantities such as
torque, speed, position, energy, and power. In a permanent-magnet motor one of
these fields is produced by permanent magnets and the other is produced by
currents. The mechanism that contains the permanent magnets is called the
\emph{rotor} because it is free to rotate, while the one that carries the currents
is called the \emph{stator} because it is generally stationary relative to the
labratory.

\section{Preliminaries}

\subsection{Maxwell's Equations}
Begin with the macroscopic Maxwell equations in SI units,

\begin{align}
\nabla \cdot \mathbf{B} &= 0,\\
\nabla \times \mathbf{E} &= -\frac{\partial \mathbf{B}}{\partial t},\\
\nabla \cdot \mathbf{D} &= \rho_\mathrm{f},\\
\nabla \times \mathbf{H} &= \mathbf{J}_\mathrm{f} + 
\frac{\partial\mathbf{D}}{\partial t},
\end{align}
where $\rho_\mathrm{f}$ and $\mathbf{J}_\mathrm{f}$ denote the free charge and
free current densities, and
\[
\mathbf{B} = \mu_0(\mathbf{H} + \mathbf{M}),\qquad
\mathbf{D} = \varepsilon_0 \mathbf{E} + \mathbf{P}
\]
are the magnetic flux density and electric displacement expressed in terms of
the magnetic and electric polarizations $\mathbf{M}$ and $\mathbf{P}$.

\subsection{Magnetostatic Approximation}
The finite propogation of electromagnetic disturbances in the air-gap is
$c = 1/\sqrt{mu_0\varepsilon_0} \approx 3 \times 10^8\, \mathrm{m/s}$.
Let $g$ denote the characteristic thickness of the air-gap (typically of the 
order of $1\mathrm{mm}$) and let $f_e$ be the electrical excitation frequency of
the motor. The time required for an electromagnetic distrubance to cross the air-
gap is 
\[
\tau = \frac{1}{f_e}
\]
The ratio of these time scales is
\[
\varepsilon_t = \frac{\tau_{\mathrm{em}}}{T_e} = \frac{g f_e}{c}
\]
Even for conservative values such as $g = 1\,\mathrm{mm}$ and $f_e = 10\,\mathrm{kHz}$
one obtains 
\[
\varepsilon_t = \frac{10^{-3}\times 10^4}{3\times 10^8}\approx 3 \times 10^{-8}
\]
and for typical permanent-magnet motors operating in the $f_e \lesssim 10^3\,\mathrm{Hz}$
the ratio is several orders of magnitude smaller. Thus the electromagnetic field
configuration across the air-gap adjusts effectively instantaneously relative to
the mechanical and electrical time scales of the machine.

This seperation of time scales can also be expressed directly in Maxwell's
equations. In the current-free, homogenous air-gap the magnetic field satisfies
the homogenous wave equation
\[
\nabla^2 \mathbf{H} - 
\mu_0\varepsilon_0 \frac{\partial^2 \mathbf{H}}{\partial t^2} = 0
\]
Introducing characteristic length $L$ (here $L \sim g$) and angular frequency
$\omega = 2 \pi f_e$, and dimensionles variables
$\tilde{\mathbf{x}} = \mathbf{x}/L$ and $\tilde{t} = \omega t$, this equation 
takes the form
\[
\tilde{\nabla}^2\mathbf{H} - \left(\frac{\omega L}{c}\right)^2
\frac{\partial^2 \mathbf{H}}{\partial \tilde{t}^2} = 0.
\]
The dimensionless parameter 
\[
\varepsilon_\omega = \frac{\omega L}{c} = \frac{2\pi f_e L }{c}
\]
measures the importance of wave-propogation effects. For the dimensions and 
frequencies of interest in rotating electrical machines $\varepsilon_\omega \ll 1$
and therefore the wave term can be neglected. The field in the air-gap then
satisfies the magnetostatic equation
\[
\nabla \times \mathbf{H} \approx \mathbf{0}, \qquad 
\nabla \cdot \mathbf{B} = 0,
\]
with $\mathbf{B} = \mu_0 \mathbf{H}$.

\subsection{Air-gap}
The air-gap region $V_g$ is the annular region of space between the rotor and 
stator surfaces, as shown in Figure~\ref{fig:air-gap}.  On the macroscopic level
of Maxwell's equations, material effects enter through the polarization $\mathbf{P}$
and magnetization $\mathbf{M}$, which relate to the fields via
\[
\mathbf{D} = \varepsilon_0 \mathbf{E} + \mathbf{P}, \qquad 
\mathbf{B} = \mu_0(\mathbf{H} + \mathbf{M}).
\]
Free charges and currents are described by the free charge density $\rho_mathrm{f}$
and free current density $\mathbf{J}_\mathrm{f}$. Bound charges and currents 
associated with materials are encoded in $\mathbf{P}$ and $\mathbf{M}$, in particular,
the bound current density is $\mathbf{J}_\mathrm{b} = \nabla \times \mathbf{M}$ 
and the total macroscopic current density is $\mathbf{J}_\mathrm{f} + \mathrm{J}_\mathrm{b}$.

In electric machines, the sources of magnetic fields are
\begin{itemize}
    \item free currents flowing in the stator,
    \item bound currents associated with magnetized materials in the rotor's
          permanent magnets.
\end{itemize}
By construction, all current carrying conductors are embedded in the stator and 
do not cross the air-gap. Likewise, all magnetized materials are confined to the
rotor and do not cross the air-gap. The air-gap $V_g$ contains only air and
consequently
\[
\mathbf{J}_\mathrm{f} = \mathbf{0},\qquad \mathbf{M} = \mathbf{0}\quad
\text{in }V_g.
\]

The medium filling the gap is air at standard conditions. On the macroscopic
scale relevant to machines, air is extremely weakly polarizable and non-magnetizable,
with relative permittivity and permeability very close to linear. It is therefore
an excellent approximation to treat the air-gap as a linear, homogenous, isotropic
medium with
\[
\mathbf{P} \approx \mathbf{0}, \qquad \mathbf{M}\approx \mathbf{0},
\]
so that the constitutive relations reduce to 
\[
\mathbf{B} = \mu_0 \mathbf{H}, \qquad
\mathbf{D} = \varepsilon_0 \mathbf{E}
\quad \text{in } V_g
\]
In other words, within the air-gap the electromagnetic field behaves as in 
vacuum: there are no material sources, and all sources are effectively represented
by boundary conditions on the enclosing rotor and stator surfaces.

\subsection{Electromagnetic force, momentum, and the Maxwell stress tensor}
A rigid rotor experiences torque because electromagnetic fields exert mechanical
forces on the charges and currents it contains.  At the macroscopic level this
mechanical force density is given by the Lorentz expansion
\begin{equation}
\mathbf{f} = \rho\,\mathbf{E} + \mathbf{J}\times\mathbf{B},
\label{eq:lorentz-force-density}
\end{equation}
where $\rho$ and $\mathbf{J}$ are the total charge and current densities 
(with bound and free contributions). The total force on a body occupying a
volume $V$ is therefore
\begin{equation}
\mathbf{F} = \int_V \mathbf{f}\, dV,
\label{eq:force-volume-integral}
\end{equation}
and the total torque about the origin is
\begin{equation}
\boldsymbol{\tau} = \int_V \mathbf{r}\times\mathbf{f}\, dV.
\label{eq:torque-volume-integral}
\end{equation}
For a rotating electrical machine this volume would include the detailed
distributions of currents and materials in the rotor and stator. Evaluating
the \eqref{eq:force-volume-integral} -- \eqref{eq:torque-volume-integral} 
directly in terms of $\rho$ and $\mathbf{J}$ is inconvenient and obscures the 
fact that the net torque depends only on the electromagnetic field in the air-gap.

A more useful representation of the mechanical action of the field is obtained
by recasting the Lorentz force density in terms of the electromagnetic field
itself. Combining \eqref{eq:lorentz-force-density} with Maxwell's equations and
using standard vector identities one finds the local momentum-balance relation
\begin{equation}
\mathbf{f} = \nabla \cdot \mathbf{T}
- \frac{\partial \mathbf{g}}{\partial t}
\label{eq:momentum-balance}
\end{equation}
where
\begin{equation}
\mathbf{g} = \varepsilon_0\, \mathbf{E} \times \mathbf{B}
\label{eq:field-momentum-density}
\end{equation}
is the electromagnetic momentum density and $\mathbf{T}$ is the \emph{Maxwell
stress tensor} with Cartesian components
\begin{equation}
T_{ij} = \varepsilon_0\Bigl(E_i E_j - \tfrac{1}{2}\delta_{ij} E^2\Bigr)
+ \frac{1}{\mu_0}\Bigl(B_i B_j - \tfrac{1}{2}\delta_{ij}B^2\Bigr),
\qquad i, j \in \{x, y, z\}
\label{eq:maxwell-stress-general}
\end{equation}
Equation \eqref{eq:momentum-balance} expresses conservation of total momentum:
the Lorentz force density $\mathbf{f}$ acting on matter is balanced by the
divergence of the field stress and the time rate of change of the electromagnetic
momentum.

Integrating \eqref{eq:momentum-balance} over a volume $V$ and applying the
divergence theorem gives
\begin{equation}
\mathbf{F} = \int_V \mathbf{f}\, dV
= \oint_{\partial V} \mathbf{T} \cdot \mathbf{n}\, dS - 
\frac{d}{dt}\int_V \mathbf{g}\, dV,
\label{eq:force-surface-integral}
\end{equation}
where $\partial V$ is the boundary surface of $V$ with outward unit normal 
$\mathbf{n}$. Thus the net mechanical force on the material inside $V$ can
be computed directly from the electromagnetic field on the boundary of $V$.
A similar manipulation applied to $\eqref{eq:torque-volume-integral}$ yields
\begin{equation}
  \boldsymbol{\tau} =
  \oint_{\partial V} \mathbf{r}\times (\mathbf{T}\cdot\mathbf{n})\, dS
  -\frac{d}{dt}\int_V\mathbf{r}\times\mathbf{g}\, dV.
  \label{eq:torque-surface-integral}
\end{equation}
The vector $\mathbf{T}\cdot\mathbf{n}$ is the traction exerted by the field on
a surface element with normal $\mathbf{n}$: its normal component corresponds to
an electromagnetic pressure and its tangential components correspond to shear
stresses.

In the magnetoquasistatic operating regime of a permanent-magnet motor, the
fields in the air-gap vary slowly and the electric field is small compared with 
the magnetic field insofar as mechanical effects are concerned. To an excellent
approximation in the air-gap, the electric contribution to the stress tensor 
and field-momentum term may be neglected, so that 
\eqref{eq:maxwell-stress-general} reduces
\begin{equation}
T_{ij} = \frac{1}{\mu_0}\Bigl(B_i B_j - \tfrac{1}{2}\delta_{ij} B^2\Bigr),
\label{eq:maxwell-stress-magnetic}
\end{equation}
and the torque on the material inside $V$ can be obtained from the purely
magnetic surface integral
\begin{equation}
  \boldsymbol{\tau}
  = \oint_{\partial V} \mathbf{r} \times (\mathbf{T}\cdot\mathbf{n})\, dS.
  \label{eq:torque-magnetic-stress}
\end{equation}

The key advantage of \eqref{eq:torque-magnetic-stress} in the motor context is
that $\partial V$ may be taken to lie entirely in the current-free,
magnetization-free air-gap.  All the details of the stator and rotor implementation
enter only through the magnetic field $\mathbf{B}$ on this surface.  The 
Maxwell stress tensor therefore provides a rigorous way to compute the net 
electromagnetic torque on the rotor using only the air-gap field.

For the machines considered here the geometry is ideally invariant in the axial
$z$-direction and only torque about the $z$-axis is of interest.  A convenient
choice for $\partial V$ is a cylindrical surface of radius $R$ in the air-gap,
coaxial with the rotor and extending over the stack length $L$. On this surface
the outward normal is $\mathbf{n} = \hat{\mathbf{e}}_r$ and the traction is
\[
\mathbf{T}\cdot\mathbf{n} = 
T_{rr} \hat{\mathbf{e}}_r + T_{\theta r} \hat{\boldsymbol{\theta}} +
T_{zr} \hat{\mathbf{e}}_z
\]
The torque density about the $z$-axis is determined by the tangential component
of this traction:
\[
d\tau_z = R\, T_{\theta r}\, dS,
\]
with $dS = R L\, d\theta$ for the cylndrical surface element.  In the 
two-dimensional model used here the fields are independent of $z$ and $B_z\approx 0$,
so that \eqref{eq:maxwell-stress-magnetic} gives
\begin{equation}
T_{\theta r} = \frac{1}{\mu_0} B_\theta B_r.
\label{eq:t-theta-r}
\end{equation}
Integrating around the circumference yields the electromagnetic torque about the
$z$-axis,
\begin{equation}
  \tau_z
= \frac{R^2 L}{\mu_0} \int_0^{2\pi} B_r(\theta)\, B_\theta(\theta)\, d\theta,
\label{eq:torque-Br-Btheta}
\end{equation}
which will serve as the starting point for the harmonic analysis in the following
sections.  This expression shows explicitly that the torque on the rotor is
goverend by the interaction between the radial and tangential components of the
magnetic field in the air-gap.

\begin{figure}[h!]
    \centering
    \includegraphics[width=\textwidth]{plots/img/air_gap.png}
    \caption{The air-gap of a motor. The region of space where the stator and r
    otor fields interact to produce torque on the surface of the rotor. No
    fields are shown.}
    \label{fig:air-gap}
\end{figure}


\section{Solving Maxwell's Equations in the Air-Gap}
\subsection{Scalar Magnetic Potential and Laplace's Equation}
Within the air-gap rergion $V_g$ there are no free conduction currents and no
magnetized materials, and the medium is well approximated as a linear, homogenous
and isotropic. In the magnetoquasistatic regime this implies that Maxwell's equations
reduce in $V_g$ to
\begin{equation}
\nabla \times \mathbf{H} = \mathbf{0}, \qquad
\nabla \cdot \mathbf{B} = 0, \qquad \mathbf{B} = \mu_0 \mathbf{H}
\label{eq:magnetostatic-gap}
\end{equation}
The first of these shows that $\mathbf{H}$ is irrotational in the air-gap.
Hence, on any simple-connected subregion of $V_g$ there exists a scalar magnetic
potential $\Phi$ such that
\begin{equation}
\mathbf{H} = -\nabla\Phi \qquad \Rightarrow \qquad \mathbf{B} = -\mu_0\nabla\Phi.
\label{eq:B-from-Phi-gap}
\end{equation}
\begin{equation}
\nabla^2 \Phi = 0 \qquad \text{in } V_g,
\label{eq:laplace-gap}
\end{equation}
so the scalar magnetic potential is harmonic in the air-gap.

As previously discussed, in the machines considered here the fields are approximetly
invariant along the axial direction $z$, so the problem reduces to two dimensions
in the cross-sectional $(r,\theta)$ plane.  In cylindrical coordinates with no
$z$-dependence, the Laplace equation \eqref{eq:laplace-gap} becomes
\begin{equation}
\frac{\partial^2 \Phi}{\partial r^2}
+ \frac{1}{r}\frac{\partial \Phi}{\partial \theta^2} = 0.
\label{eq:laplace-polar}
\end{equation}
On the annular ring $R_s < r < R_r$ the general seperable solution of 
\eqref{eq:laplace-polar} can be written as
\begin{equation}
  \Phi(r,\theta) = A_0 + B_0\ln r + \sum_{n=1}^\infty \Bigl[
    \bigl(a_n r^n + b_n r^{-n}\bigr)\cos(n\theta) + \bigl(c_n r^n + d_n r^{-n}\bigr)
    \sin(n\theta)\Bigr],
\label{eq:Phi-general}
\end{equation}
where the coefficients are determined by the boundary conditions imposed by the
rotor and stator on $r=R_s$ and $r=R_r$.  The constant $A_0$ is an arbitary gauge
for the potential and does not affect the magnetic field.  The $B_0\ln r$ term
and the $r^{\pm n}$ terms represent different radial behaviours compatible with
Laplace's equation on the finite annulus; the particular combination present in
a given machine is fixed by the physical boundary conditions.

The magnetic flux density components in cylindrical coordinates are obtained
from \eqref{eq:B-from-Phi-gap} as
\begin{equation}
B_r(r,\theta) = -\mu_0 \frac{\partial \Phi}{\partial r}, \qquad
B_\theta(r, \theta) = -\frac{\mu_0}{r}\frac{\partial \Phi}{\partial \theta}.
\label{eq:BrBtheta-from-Phi}
\end{equation}
For the torque calculation it is sufficient to know $B_r$ and $B_\theta$ on a
single cylndrical surface of radius $R$ lying in the air-gap. At any fixed
radius $R$ the functions $B_r(R,\theta)$ and $B_\theta(R,\theta)$ are
$2\pi$-periodic in $\theta$ and therefore admit a Fourier series of the form
\begin{equation}
B_r(R,\theta) = \sum_{n=1}^\infty \bigl(a_n \cos(n\theta) + b_n\sin(n\theta)\bigr),
\end{equation}

\subsection{Ideal Rotor and Stator Fields}
Because the magnetostatic equations in the air-gap are linear, the total magnetic 
field can be decomposed into contributions from the rotor and stator sources.
Let $\mathbf{B}^r$ denote the field produced by the rotor alone with the
stator source currents set to zero, and let $\mathbf{B}^2$ denote the field 
produced by the stator currents with the rotor demagnetized.  In the air-gap
$V_g$ both fields satisfy the same governing equations,
\begin{equation}
\nabla\times\mathbf{H}^{r,s} = \mathbf{0},\qquad
\nabla\cdot\mathbf{B}^{r,s} = 0,\qquad
\mathbf{B}^{r,s} = \mu_0 \mathbf{H}^{r,s},
\end{equation}
and superposition gives
\begin{equation}
\mathbf{B} = \mathbf{B}^r + \mathbf{B}^s,\qquad
\mathbf{H} = \mathbf{H}^r + \mathbf{H}^s.
\end{equation}
The distinct role of the rotor and stator enter through their boundary conditions 
at the rotor surface $r=R_r$ and stator surfaces $r=R_s$, respectively.
These boundary conditions are determined by the internal magnetization and currents
in the rotor and stator regions.

At any material interface the fields satisfy
\begin{align}
  \hat{\mathbf{n}}\cdot(\mathbf{B}_2 - \mathbf{B}_1) &= 0, \label {eq:BC-normalB}\\
  \hat{\mathbf{n}}\times(\mathbf{H}_2 -\mathbf{H}_1) &= \mathbf{H}_\mathrm{f},
  \label{eq:BC-tangH}
\end{align}
where $\hat{\mathbf{n}}$ is the unit normal pointing from region $1$ into region
$2$ and $\mathbf{K}_\mathrm{f}$ is any free surface current density on the interface.
Equation \eqref{eq:BC-normalB} expresses continuity of the normal component of
$\mathbf{B}$, while \eqref{eq:BC-tangH} relates the jump in tangential $\mathbf{H}$
to surface currents. In a typical machine the rotor and stator iron have relative
permeability $\mu_r \gg 1$, so that, to a good approximation, $\mathbf{B}$ is nearly
normal to the iron surfaces and the normal component is determined by the internal
magnetization or current distribution.

\subsection{Ideal Rotor}
Consider the rotor assembly occupying $r < R_r$, consisting of high-permeability
iron and permanent magnets. With the stator unexcited, the rotor field $\mathbf{B}^r$
in the air-gap is governed by the magnetostatic equations with boundary conditions on
$r = R_r$ fixed by the rotor magnetization.  An \emph{ideal rotor of spatial order
$n$} is defined as a rotor for which, in this rotor-only configuration, the radial
and tangential components of the air-gap field at the surface of the rotor satisfy
\begin{equation}
  B_r^r(R_r,\theta) = a \cos(n\theta),\qquad
  B_\theta^r(R_r, \theta) = 0,
  \label{eq:ideal-rotor-BC}
\end{equation}
for some real amplitude $a > 0$ and integer $n \ge 1$.

This definition encodes two simplifying properties:
\begin{enumerate}
  \item The radial flux on the rotor surface is a \emph{single spatial harmonic}
        of order $n$ (no higher-order harmonics are present).
  \item The field at the rotor surface is \emph{purely radial} (normal to the
        surface), consistent with the limit of very high rotor permeability in
        which the tangential component of $\mathbf{B}$ at the iron surface is 
        supressed.
\end{enumerate}
Physically, such a field can be realized arbitrarily closely by an appropriate
circumferential magnetization pattern in a cylindrical permanent-magnet shell
backed by a high permeability iron, as illustrated in Figure \ref{fig:rotor}.
The constant $n$ is the number of \emph{pole pairs} $p$ of the rotor field.

\subsection{Ideal Stator}
Similarly, consider the stator assembly occuping $r > R_s$, consisting of 
ferromagnetic iron and current-carrying windings. With the rotor demagnetized,
the stator field $\mathbf{B}^s$ in the air-gap is determined by Ampere's Law
and the current distribution in the windings. The tangential component of $\mathbf{H}^s$
at the stator inner surface is related to the stator surface current density
$\mathbf{K}_\mathrm{f}$ by \eqref{eq:BC-tangH}. In many machines the stator is
designed so that the resulting air-gap field is dominated by a single spatial
harmonic of perscribed order.

An \emph{ideal stator of spatial order $n$} is defined as a stator for which,
in the stator-only configuration, the air-gap field at the stator surface satisfied
\begin{equation}
B_r^s(R_s,\theta) = 0, \qquad 
B_\theta^s(R_s,\theta) = b\cos(n\theta - \phi_s),
\label{eq:ideal-stator-BC}
\end{equation}
for some amplitude $b>0$, phase $\phi_s$, and the same integer $n \ge 1$.
Here 
\begin{enumerate}
  \item The tangential flux density on the stator surface is a \emph{single
  spatial harmonic} of order $n$, corresponding to a sinusoidally distributed 
  stator magnetomotive force (MMF) with $n$ pole pairs.
  \item The field at the stator surface is \emph{purely tangential}, as expected
  in the ideal limit of a smooth, high-permeability stator core with no normal
  flux crossing the inner surface except through discrete teeth.
\end{enumerate}
In practice, a distributed stator winding with appropriately chosen coil pitch
and distribution can approximate \eqref{eq:ideal-stator-BC} closely by supressing
higher spatial harmonics and producing a nearly sinusoidal rotating air-gap field.

\subsection{Ideal Rotor-Stator Pair and Torque}
When both the rotor and stator are present and energized, the total air-gap field
is the sum
\[
B_r = B_r^r + B_r^s, \qquad
B_\theta = B_\theta^r + B_\theta^s.
\]
For an ideal rotor--stator pair of order $n$, as defined above, the dominant
contribution to the torque arises from the interaction of the $n$th spatial
harmonic of $B_r$ and $B_\theta$. Evaluated on any cylindrical surface 
$R_r < R < R_s$ in the air-gap, these harmonics inherit the same spatial order
$n$ and their amplitudes and phase can be written as
\[
B_r(R,\theta) \approx B_{r,n}\cos(n\theta - \phi_r),\qquad
B_\theta(R,\theta)\approx B_{\theta,n}\cos(n\theta - \phi_\theta),
\]
where $B_{r,n}$ and $B_{\theta,n}$ are proportional to $a$ and $b$ and $\phi_r$
and $\phi_\theta$ are the corresponding phase angles. Substituting these into
the general torque expression
\[
T = \frac{\pi R^2 L}{\mu_0} \sum_{k=1}^\infty B_{r,k} B_{\theta,k} \cos\bigl(
  \phi_{r,k} - \phi_{\theta,k}\bigr),
\]
For a perfectly aligned ideal rotor--stator pair we choose the phase reference
such that $\phi_r = \phi_\theta$, so that the $n$th-order contribution is maximal
and
\begin{equation}
T_\mathrm{max} = \frac{\pi R^2 L}{\mu_0} B_{r,n} B_{\theta, n}.
\end{equation}
In particular, evaluated on the rotor surface $R = R_r$ for the boundary conditions 
\eqref{eq:ideal-rotor-BC}--\eqref{eq:ideal-stator-BC} with $\phi_s=0$, one has
$B_{r,n}=a$ and $B_{\theta,n}=b$ and therefore
\begin{equation}
  T_\mathrm{max} = \frac{\pi R^2 L}{\mu_0}\, a b.
  \label{eq:T-ideal-ab}
\end{equation}
The ideal rotor and stator thus provide a useful reference: for a given air-gap
radius and stack length, they define the maximum torque obtainable from a single
spatial harmonic of radial and tangential flux density in the air-gap.


\begin{figure}[h!]
    \centering
    \includegraphics[width=0.5\textwidth]{plots/img/rotor_analytic.png}
    \caption{The magnetization of an ideal permanent magnet rotor.}
    \label{fig:rotor}
\end{figure}

\begin{figure}[h!]
    \centering
    \includegraphics[width=\textwidth]{plots/img/rotor_analytic_magnetization.png}
    \caption{The magnetic field outside an ideal permanent magnet rotor.}
    \label{fig:rotor_magnetization}
\end{figure}

\begin{figure}[h!]
    \centering
    \includegraphics[width=\textwidth]{plots/img/rotor_and_stator_air_gap.png}
    \caption{The magnetic field of an ideal stator and rotor decomposing the
    field in term of the component contributions from the stator and rotor.
    The stator field is entirely tangential, while the rotor field is entirely
    radial at the center of the magnetic poles.}
    \label{fig:rotor-stator-air-gap}
\end{figure}

\begin{figure}[h!]
    \centering
    \includegraphics[width=\textwidth]{plots/img/rotor_and_stator_torque_density.png}
    \caption{The tangential shear component of the Maxwell stress tensor in the
    air-gap. This gives the torque density at each point. The torque denisty is
    highest along the surface of the rotor.}
    \label{fig:rotor-stator-torque-density}
\end{figure}

\end{document}