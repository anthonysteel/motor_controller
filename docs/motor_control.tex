\documentclass[11pt]{article}

% Encoding and language
\usepackage[utf8]{inputenc}
\usepackage[T1]{fontenc}
\usepackage[english]{babel}

% Math and symbols
\usepackage{amsmath, 
            amssymb,
            graphicx,
            subcaption}
\usepackage[europeanresistors,americaninductors]{circuitikz}

% Better spacing and layout
\usepackage{microtype}
\usepackage{geometry}
\geometry{margin=1in}

% Hyperlinks
\usepackage[colorlinks=true, linkcolor=blue, urlcolor=blue, citecolor=blue]{hyperref}

\title{Permanent Magnet Motor Control}
\author{Anthony J. Steel}
\date{\today}

\begin{document}

\maketitle
The fundamental unit of a permanent magnet motor is the rotor containing
$p$ pairs of ideal radial-flux permanent magnets as shown in Figure \ref{fig:rotor}.
The rotor is a very symmetrical object and this symmetry should be 
exploited immediately in its mathematical description.
\begin{figure}[h!]
    \centering
    \includegraphics[width=0.35\textwidth]{graphs/img/rotor.png}
    \caption{An ideal rotor with 12 pairs radial-flux permanet magnets.}
    \label{fig:rotor}
\end{figure}
Start by linearizing the rotor in terms of its axial angle as shown in Figure 
\ref{fig:linearized-rotor-mechanical}. This angle is called the 
\emph{mechanical angle} and it is the angle that would be measured by the rotation
of a shaft rigidly fixed to the center of the rotor.  The rotor is clearly periodic
every $2\pi$ mechanical radians. However, there is more fundamental periodicity
defined by the number of $p$ of pole pairs as shown in Figure 
\ref{fig:linearized-rotor-electrical}.
\begin{figure}[h!]
    \centering
    \includegraphics[width=0.75\textwidth]{graphs/img/rotor_unrolled.png}
    \caption{A linearized rotor expressed in terms of the mechanical angle $\theta_m$.}
    \label{fig:linearized-rotor-mechanical}
\end{figure}
This angle is called the \emph{electrical angle} and it is fundamental because 
it is a universal coordinate for all rotors. For reasons that will become clearl
later, it is mathematically convienient to choose the origin of the coordinate system
to be centered in the middle of one of the magnets.
\begin{figure}[h!]
    \centering
    \includegraphics[width=0.75\textwidth]{graphs/img/rotor_unrolled_electrical_angle.png}
    \caption{A linearized rotor expressed in terms of the electrical angle $\theta_e$.
    This is the fundamental magnetic unit of the rotor.}
    \label{fig:linearized-rotor-electrical}
\end{figure}

The ideal radial-flux permanent magnets on the rotor are perfect magnetic dipoles
and their magnetic field is found by solving Maxwell's equations numerically.
The resulting field is shown in Figure \ref{fig:rotor-magnetization}.
\begin{figure}[h!]
    \centering
    \includegraphics[width=\textwidth]{graphs/img/rotor_magnetization.png}
    \caption{The magnetic field of one electrical period of a rotor with
    ideal radially polarized magnets.}
    \label{fig:rotor-magnetization}
\end{figure}
It is known from Maxwell's equations that currents produce magnetic fields.
Therefore, by controlling currents near the surface of the rotor it will be
possible to generate an interaction between two fields to produce rotation.
The purpose of motor control is to control the electromagnetic properties of the 
interaction between the fields of the rotor and the currents.

In order to understand how the current field acts on the rotor, it must be
understood how electromagnetic fields exert forces on material bodies.
The fundamental principle is that electromagnetic forces arise from the 
momentum carried by the electromagnetic field itself.  This momentum flux is
described by the \emph{Maxwell stress tensor} which encodes the pressure and 
shear stresses exerted by the field on surfaces that couple to it.

In quasi-magnetostatic conditions relevant to the interaction of the rotor
and stator, the Maxwell stress tensor $T_{ij}$ is 
\[
T_{ij} = \frac{1}{\mu_0}\Big(B_iB_j - \frac{1}{2}\delta_{ij}B^2) \text{ where $i,j=\{\theta_e, r\}$}.
\]
The indices of this tensor have geometric meaning. $j$ specifies the direction 
of the surface normal that the electromagnetic field is acting on. $i$ specifies
the direction of the component of the force acting on the surface.
The surface that a force will applied to is the surface of the rotor whose unit-normal 
is $\hat{r}$ and the component of the force that will produce torque will be in the
tangential $\hat{\theta}$ direction. Therefore the \emph{only} component of the
Maxwell stress tensor that will produce torque on the rotor is the $T_{\theta r}$ term.

\begin{figure}[h!]
    \centering
    \includegraphics[width=\textwidth]{graphs/img/rotor_maxwell_tensor.png}
    \caption{The magnetic field of one electrical period of a rotor with
    ideal radially polarized magnets.}
    \label{fig:rotor_maxwell_tensor}
\end{figure}


\end{document}